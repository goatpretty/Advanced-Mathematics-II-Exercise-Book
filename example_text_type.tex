% !TeX root = example_text_type.tex

% 使用 ExBook 文档类,并传递选项
\documentclass[standard]{ExBook} 
 
\begin{document}

% 加载配置  
% 封面设置
\CoverImg{img/cover.jpg} % 封面图片
\PreTitle{ExBook · 刷题本} % 前置标题
\Title{高等数学(下)} % 主标题
\TitleDescription{刷~题~集} % 副标题
\TypeOne{A4紧凑版} % A4紧凑版下的类型标识
\TypeTwo{A4标准版} % A4标准版下的类型标识
\TypeThree{横版Pad版} % 横版Pad版下的类型标识
\TypeFour{A4宽松版} % A4宽松版下的类型标识
\TypeFive{A4单题版} % A4单题版下的类型标识
\TypeSix{竖版Pad版} % 竖版Pad版下的类型标识
\motto{不处对象~喵喵喵} % 封面座右铭
\Creator{GoatPretty} % 制作人
\UpdateTime{\today} % 更新时间
\OnlineCheckUrl{https://github.com/ExBook/ExBook} % 在线勘误文档地址

% 页眉页脚设置
\Lhead{高等数学(下)} % 左页眉 
\Chead{} % 中页眉、平板模式(padl或padp)下页眉中间的文字
\Rhead{习题集} % 右页眉、平板模式(padl或padp)下页眉右侧的文字
\LheadC{GoatPretty} % 平板模式(padl或padp)下页眉左侧的文字

% 水印设置
\TextWater{【GoatPretty】} % 行内文字水印 
\WaterImg{img/water.png} % 图片水印 出现在页面的右下角

% 主题颜色设置,可选主题有:
% blue (默认)
% green
% purple
% orange
% infj
% enfp
% infp
% esfp
% intj
% entp
% isfj
% enfj
\setThemeColor{\green}



% 加载封面
\maketitle 
 

\setcounter{page}{1}
\tableofcontents 
    
\clearpage 

\section{向量代数与空间解析几何}

\subsection{向量及其线性运算}
\qanswerloc{20}
\begin{qitems}

    \begin{bbox}
        \bigq
        \qitem 已知向量 $\overrightarrow{OA}$ 的模为 8,且它与 $ox$ 轴和 $oy$ 轴的夹角均为 $\dfrac{\pi}{3}$,求 $\overrightarrow{OA}$ 的坐标表示式。
    \end{bbox}

    \begin{bbox}
        \bigq
        \qitem 已知三点 $A\left( {1,0,4}\right)$, $B\left( {3,2,2}\right)$, $C\left( {-2,-1,0}\right)$,$D$ 为 $AB$ 的中点,求与 $\overrightarrow{CD}$ 平行的单位向量。
    \end{bbox}

    \begin{bbox}
        \qitem 已知 $A\left( {1,2,0}\right)$、$B\left( {2,-1,3}\right)$,求:
        \begin{subqitems}
            \subqitem 向量 $\overrightarrow{AB}$ 在三个坐标轴上的投影;
            \subqitem 向量 $\overrightarrow{AB}$ 的模;
            \subqitem 向量 $\overrightarrow{AB}$ 的方向余弦;
            \subqitem 与向量 $\overrightarrow{AB}$ 方向一致的单位向量。
        \end{subqitems}
    \end{bbox}

    \begin{bbox}
        \bigq
        \qitem 设 $\overrightarrow{a}=\left( {4,5,-3}\right)$,$\overrightarrow{b}=\left( {1,3,6}\right)$,问实数 $\lambda,\mu$ 满足什么条件时,可使 $\lambda \overrightarrow{a}+\mu \overrightarrow{b}$ 与 $z$ 轴垂直?
    \end{bbox}

\end{qitems}

\subsection{数量积 向量积 混合积}
\begin{qitems}

    \begin{bbox}
        \qitem 向量 $\vec{a}=3\vec{i}+2\vec{j}+\vec{k}$ 与 $\vec{b}=2\vec{i}-3\vec{j}$ 的位置关系是\blankbox。
        \fourchoices{平行}{垂直}{相交}{以上都不是}
    \end{bbox}

    \begin{bbox}
        \qitem 设三个向量 $\vec{a},\vec{b},\vec{c}$ 满足关系式 $\vec{a}+\vec{b}+\vec{c}=0$,则 $\vec{a}\times\vec{b}=$\blankbox。
        \fourchoices{$\vec{c}\times\vec{b}$}{$\vec{b}\times\vec{c}$}{$\vec{a}\times\vec{c}$}{$\vec{b}\times\vec{a}$}
    \end{bbox}

    \begin{bbox}
        \qitem 已知 $\overrightarrow{OA}=\overrightarrow{i}+3\overrightarrow{k}$,$\overrightarrow{OB}=\overrightarrow{j}+3\overrightarrow{k}$,则 $\bigtriangleup OAB$ 的面积为\blankbox。
        \vspace{2mm} 
        \fourchoices{$19$}{$\dfrac{1}{2}\sqrt{19}$}{$\sqrt{19}$}{$29$}
    \end{bbox}

    \begin{bbox}
        \qitem 非零向量 $\vec{a},\vec{b},\vec{c}$ 共面的充分必要条件是\blankbox。
        \fourchoices
        {$\vec{a}\cdot\left( \vec{b}\times\vec{c}\right)=0$}
        {$\vec{a}\cdot\left( \vec{b}\cdot\vec{c}\right)=0$}
        {$\vec{a}\times\left( \vec{b}\times\vec{c}\right)=0$}
        {$\vec{a}\times\left( \vec{b}\cdot\vec{c}\right)=0$}
    \end{bbox}

    \begin{bbox}
        \qitem 已知 $\vec{a}=\left( {1,1,-4}\right)$,$\vec{b}=\left( {2,0,-2}\right)$,求:
        \begin{subqitems}
            \subqitem $\left( \vec{a}-\vec{b}\right)\cdot\vec{a}$;
            \subqitem $\left( \vec{a}\times\vec{b}\right)\cdot\left( \vec{a}+\vec{b}\right)$。
        \end{subqitems}
    \end{bbox}

    \begin{bbox}
        \bigq
        \qitem 已知 $\vec{a}=\left( {1,1,-4}\right)$,$\vec{b}=\left( {2,0,-2}\right)$,求 ${\operatorname{Prj}}_{\vec{a}}\vec{b}$。
    \end{bbox}

    \begin{bbox}
        \bigq
        \qitem 已知 $\left|\vec{a}\right|=1$,$\left|\vec{b}\right|=2$,$\left( \vec{a},\vec{b}\right)=60^{\circ}$,求 $\left| \vec{a}-\dfrac{1}{3}\left( \vec{a}-\vec{b}\right)\right|$。
    \end{bbox}

\end{qitems}

\subsection{曲面及其方程}
\begin{qitems}

    \begin{bbox}
        \qitem 方程 $-\dfrac{x^{2}}{4}+\dfrac{y^{2}}{9}=1$ 表示的空间曲面是\blankline。\vspace{2mm}
    \end{bbox}

    \begin{bbox}
        \qitem 方程 $\dfrac{x^{2}}{9}+\dfrac{z^{2}}{4}=1$ 表示的空间曲面是\blankline。\vspace{2mm}
    \end{bbox}

    \begin{bbox}
        \bigq
        \qitem 建立以点 $\left( {1,3,-2}\right)$ 为球心,且通过坐标原点的球面方程。
    \end{bbox}

    \begin{bbox}
        \bigq
        \qitem 将 $xoy$ 坐标面上的双曲线 $4x^{2}-9y^{2}=36$ 分别绕 $x$ 轴及 $y$ 轴旋转一周,求所生成的旋转曲面的方程。
    \end{bbox}

\end{qitems}

\subsection{空间曲线及其方程}
\begin{qitems}

    \begin{bbox}
        \bigq
        \qitem 求曲线 $C:\left\{  \begin{matrix} x^{2}+y^{2}-z^{2}=0 \\  x-z+1=0 \end{matrix}\right.$ 关于 $xoy$ 坐标面的投影柱面方程及此曲线在 $xoy$ 坐标面的投影方程测试从厕所测试测试厕所测试测试厕所测试测试厕所测试测试厕所测试测试厕所测试测试厕所测试测试厕所测试测试厕所测试测试厕所测试测试厕所测试测试厕所测试测试厕所测试测试厕所测试测试厕所测试测试厕所测试测试厕所测试测试厕所测试测试厕所测试测试厕所测试测试厕所测试测试厕所测试测试厕所测试测试厕所测试测试厕所测试测试厕所测试测试厕所测试测试。
    \end{bbox}

    \begin{bbox}
        \bigq
        \qitem 求母线平行于 $y$ 轴,且通过曲线 $\left\{  \begin{matrix} 2x^{2}+y^{2}+z^{2}=16 \\  x^{2}+y^{2}-z^{2}=0 \end{matrix}\right.$ 的投影柱面方程。
    \end{bbox}

\end{qitems}

\subsection{平面及其方程}
\begin{qitems}

    \begin{bbox}
        \qitem 平面 $x-y+2z-6=0$ 和 $2x+y+z-5=0$ 的夹角是\blankbox。
        \fourchoices{$\pi$}{$\dfrac{\pi}{2}$}{$\dfrac{\pi}{3}$}{$2\pi$}
    \end{bbox}

    \begin{bbox}
        \qitem 两平面 $2x-y-z=0$ 和 $x+y+z=0$ 的位置是\blankbox。
        \fourchoices{平行}{相交不垂直}{垂直}{共面}
    \end{bbox}

    \begin{bbox}
        \bigq
        \qitem 求过点 $A\left( {5,4,3}\right)$ 且在各坐标轴上的截距相等的平面方程。
    \end{bbox}

    \begin{bbox}
        \bigq
        \qitem 求平行于 $xoz$ 面且经过点 $\left( {2,-5,3}\right)$ 的平面方程。
    \end{bbox}

    \begin{bbox}
        \bigq
        \qitem 求通过 $z$ 轴和点 $\left( {-3,1,-2}\right)$ 的平面方程。
    \end{bbox}

\end{qitems}

\subsection{空间直线及其方程}
\begin{qitems}

    \begin{bbox}
        \bigq
        \qitem 求过点 $\left( {1,0,-2}\right)$ 且与平面 $3x+4y-z+6=0$ 平行,与直线 $\dfrac{x-3}{1}=\dfrac{y+2}{4}=\dfrac{z}{1}$ 垂直的直线方程。
    \end{bbox}

    \begin{bbox}
        \bigq
        \qitem 求过点 $\left( {3,2,-1}\right)$ 且与平面 $x-4z-3=0$ 及 $2x-y-5z-1=0$ 平行的直线方程。
    \end{bbox}

    \begin{bbox}
        \bigq
        \qitem 求通过平面 $x+y-z-2=0$ 与 $3x+y-z-5=0$ 的交线,且过点 $\left( {1,8,2}\right)$ 的平面方程。
    \end{bbox}

    \begin{bbox}
        \bigq
        \qitem 求点 $M\left( {1,2,-1}\right)$ 到直线 $\dfrac{x-1}{2}=\dfrac{y+1}{-1}=\dfrac{z-2}{3}$ 的距离。
    \end{bbox}

    \begin{bbox}
        \bigq
        \qitem 求点 $M\left( {1,2,3}\right)$ 到直线 $\left\{  \begin{matrix} x+y-z=1 \\  2x+z=3 \end{matrix}\right.$ 的距离。
    \end{bbox}

    \begin{bbox}
        \bigq
        \qitem 求点 $N\left( {-1,2,0}\right)$ 在平面 $x+2y-z+1=0$ 上的投影。
    \end{bbox}

    \begin{bbox}
        \bigq
        \qitem 确定 $\lambda$,使直线 $\dfrac{x-1}{1}=\dfrac{y+2}{2}=\dfrac{z-1}{\lambda}$ 垂直于平面 ${\pi}_{1}:3x+6y+3z+25=0$,并求该直线在平面 ${\pi}_{2}:x-y+z-2=0$ 上的投影直线的方程。
    \end{bbox}

\end{qitems}

\subsection{本章综合测验}
\begin{qitems}

    \begin{bbox}
        \qitem 若非零向量 $\overrightarrow{a}$ 和 $\overrightarrow{b}$ 满足 $\left| {\overrightarrow{a}-\overrightarrow{b}}\right|=\left| \overrightarrow{a}\right|+\left| \overrightarrow{b}\right|$,则\blankbox。
        \fourchoices{方向相同}{互相垂直}{方向相反}{平行}
    \end{bbox}

    \begin{bbox}
        \qitem 方程 $y^{2}+z^{2}-24x+8=0$ 表示\blankbox。
        \fourchoices{双曲柱面}{椭圆柱面}{锥面}{旋转抛物面}
    \end{bbox}

    \begin{bbox}
        \qitem 方程 $x^{2}+y^{2}+z^{2}=49$ 表示的曲面是\blankbox。
        \fourchoices{柱面}{球面}{锥面}{旋转抛物面}
    \end{bbox}

    \begin{bbox}
        \qitem 平面 $x=2z$\blankbox。
        \fourchoices{平行 $xOz$ 坐标面}{平行 $y$ 轴}{垂直 $y$ 轴}{通过 $y$ 轴}
    \end{bbox}

    \begin{bbox}
        \qitem 曲面 $x^{2}+y^{2}+z^{2}=9$ 与 $x+y=1$ 的交线在 $xoy$ 面上的投影为\blankbox。
        \fourchoices{椭圆柱面}{椭圆曲线}{两平行平面}{线段}
    \end{bbox}

    \begin{bbox}
        \qitem 直线 $L:\dfrac{x+3}{-2}=\dfrac{y+4}{-7}=\dfrac{z}{3}$ 与平面 $\pi:4x-2y-2z=3$ 的关系是\blankbox。
        \fourchoices{平行}{垂直相交}{L 在 $\pi$ 上}{相交但不垂直}
    \end{bbox}

    \begin{bbox}
        \qitem 直线 $L:\dfrac{x}{3}=\dfrac{y}{-2}=\dfrac{z}{7}$ 和平面 $3x-2y+7z=8$ 的关系是\blankbox。
        \fourchoices{平行}{垂直相交}{L 在 $\pi$ 上}{相交但不垂直}
    \end{bbox}

    \begin{bbox}
        \qitem 设直线 $\dfrac{x}{0}=\dfrac{y}{4}=\dfrac{z}{-3}$,则该直线必定\blankbox。
        \fourchoices{过原点且垂直于 $x$ 轴}{过原点且平行于 $x$ 轴}{不过原点,但垂直于 $x$ 轴}{不过原点,且不平行于 $x$ 轴}
    \end{bbox}


    \begin{bbox}
        \qitem 向量的终点在点 $B\left( {2,-1,7}\right)$,它在坐标轴上的投影依次是 $4$、$-4$、$7$,这个向量的起点 $A$ 的坐标为\blankline。\vspace{2mm}
    \end{bbox}

    \begin{bbox}
        \qitem 将 $xoz$ 坐标面上的曲线 $z^{2}=5x$ 绕 $x$ 轴旋转所生成的旋转曲面方程为\blankline。\vspace{2mm}
    \end{bbox}

    \begin{bbox}
        \qitem 过点 $\left( {2,-5,3}\right)$ 且平行于 $xoz$ 面的平面方程为\blankline。\vspace{2mm}
    \end{bbox}

    \begin{bbox}
        \qitem 过点 $\left( {2,4,-1}\right)$ 且平行于 $S=\left( {1,3,4}\right)$ 的直线方程为\blankline。\vspace{2mm}
    \end{bbox}

    \begin{bbox}
        \qitem 通过点 $M\left( {1,2,3}\right)$ 且与直线 $L:x=2+3t$,$y=2t$,$z=-1+t$ 垂直的平面方程为\newline\blankline。\vspace{2mm}
    \end{bbox}


    \begin{bbox}
        \bigq
        \qitem 已知 ${M}_{1}\left( {4,\sqrt{2},1}\right)$,${M}_{2}\left( {3,0,2}\right)$,求向量 ${M}_{1}{M}_{2}$ 的模、方向余弦和方向角。
    \end{bbox}

    \begin{bbox}
        \bigq
        \qitem 设向量 $\dot{\mathbf{r}}$ 的模是 4,它与轴 $\mathbf{u}$ 的夹角是 $60^{\circ}$,求 $\dot{\mathbf{r}}$ 在轴 $\mathbf{u}$ 上的投影。
    \end{bbox}

    \begin{bbox}
        \bigq
        \qitem 求向量 ${b}^{\prime}=i-j+3k$ 与 ${c}^{\prime}={i}^{\prime}-2j$ 的夹角余弦。
    \end{bbox}

    \begin{bbox}
        \bigq
        \qitem 设 $a=\left( {x,y,z}\right)$,$b=\left( {2,0,5}\right)$,$c=\left( {3,0,0}\right)$,问当 $x,y,z$ 取何值时,$a$ 与 $b$ 平行;取何值时 $a$ 与 $c$ 平行。
    \end{bbox}

    \begin{bbox}
        \bigq
        \qitem 已知 ${M}_{1}\left( {1,-1,2}\right)$,${M}_{2}\left( {3,3,1}\right)$,${M}_{3}\left( {3,1,3}\right)$,求与 $\overrightarrow{{M}_{1}{M}_{2}}$、$\overrightarrow{{M}_{2}{M}_{3}}$ 同时垂直的单位向量。
    \end{bbox}

    \begin{bbox}
        \bigq
        \qitem 化直线方程 $\left\{  \begin{matrix} x-y+z+5=0 \\  5x-8y+4z+36=0 \end{matrix}\right.$ 为对称式方程和参数方程。
    \end{bbox}

    \begin{bbox}
        \bigq
        \qitem 求直线 $L:\dfrac{x-1}{1}=\dfrac{y}{1}=\dfrac{z+1}{-1}$ 在平面 $\Pi:x-y+2z-1=0$ 上的投影直线 ${L}_{0}$ 的方程。
    \end{bbox}

    \begin{bbox}
        \bigq
        \qitem 试证直线 $\dfrac{x-1}{1}=\dfrac{y-1}{1}=\dfrac{z+3}{-2}$ 在平面 $x+y+z+1=0$ 上。
    \end{bbox}

\end{qitems}

\end{document}




