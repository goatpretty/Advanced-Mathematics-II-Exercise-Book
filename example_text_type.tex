% !TeX root = example_text_type.tex

% 使用 ExBook 文档类,并传递选项
\documentclass[standard]{ExBook} 
 
\begin{document}

% 加载配置  
% 封面设置
\CoverImg{img/cover.jpg} % 封面图片
\PreTitle{ExBook · 刷题本} % 前置标题
\Title{高等数学(下)} % 主标题
\TitleDescription{刷~题~集} % 副标题
\TypeOne{A4紧凑版} % A4紧凑版下的类型标识
\TypeTwo{A4标准版} % A4标准版下的类型标识
\TypeThree{横版Pad版} % 横版Pad版下的类型标识
\TypeFour{A4宽松版} % A4宽松版下的类型标识
\TypeFive{A4单题版} % A4单题版下的类型标识
\TypeSix{竖版Pad版} % 竖版Pad版下的类型标识
\motto{不处对象~喵喵喵} % 封面座右铭
\Creator{GoatPretty} % 制作人
\UpdateTime{\today} % 更新时间
\OnlineCheckUrl{https://github.com/ExBook/ExBook} % 在线勘误文档地址

% 页眉页脚设置
\Lhead{高等数学(下)} % 左页眉 
\Chead{} % 中页眉、平板模式(padl或padp)下页眉中间的文字
\Rhead{习题集} % 右页眉、平板模式(padl或padp)下页眉右侧的文字
\LheadC{GoatPretty} % 平板模式(padl或padp)下页眉左侧的文字

% 水印设置
\TextWater{【GoatPretty】} % 行内文字水印 
\WaterImg{img/water.png} % 图片水印 出现在页面的右下角

% 主题颜色设置,可选主题有:
% blue (默认)
% green
% purple
% orange
% infj
% enfp
% infp
% esfp
% intj
% entp
% isfj
% enfj
\setThemeColor{\green}



% 加载封面
\maketitle 
 

\setcounter{page}{1}
\tableofcontents 
    
\clearpage 

\section{向量代数与空间解析几何}

\subsection{向量及其线性运算}
\qanswerloc{1}
\begin{qitems}

    \begin{bbox}[big]
        \qitem 已知向量 $\overrightarrow{OA}$ 的模为 8,且它与 $ox$ 轴和 $oy$ 轴的夹角均为 $\dfrac{\pi}{3}$,求 $\overrightarrow{OA}$ 的坐标表示式。
    \end{bbox}

    \begin{bbox}[big]
        \qitem 已知三点 $A\left( {1,0,4}\right)$、$B\left( {3,2,2}\right)$、$C\left( {-2,-1,0}\right)$,$D$ 为 $AB$ 的中点,求与 $\overrightarrow{CD}$ 平行的单位向量。
    \end{bbox}

    \begin{bbox}[big]
        \qitem 已知 $A\left( {1,2,0}\right)$、$B\left( {2,-1,3}\right)$,求:
        \begin{subqitems}
            \subqitem 向量 $\overrightarrow{AB}$ 在三个坐标轴上的投影;
            \subqitem 向量 $\overrightarrow{AB}$ 的模;
            \subqitem 向量 $\overrightarrow{AB}$ 的方向余弦;
            \subqitem 与向量 $\overrightarrow{AB}$ 方向一致的单位向量。
        \end{subqitems}
    \end{bbox}

    \begin{bbox}[big]
        \qitem 设 $\overrightarrow{a}=\left( {4,5,-3}\right)$,$\overrightarrow{b}=\left( {1,3,6}\right)$,问实数 $\lambda,\mu$ 满足什么条件时,可使 $\lambda \overrightarrow{a}+\mu \overrightarrow{b}$ 与 $z$ 轴垂直?
    \end{bbox}

\end{qitems}

\subsection{数量积 向量积 混合积}
\begin{qitems}

    \begin{bbox}[choice]
        \qitem 向量 $\vec{a}=3\vec{i}+2\vec{j}+\vec{k}$ 与 $\vec{b}=2\vec{i}-3\vec{j}$ 的位置关系是\blankbox。
        \fourchoices{平行}{垂直}{相交}{以上都不是}
    \end{bbox}

    \begin{bbox}[choice]
        \qitem 设三个向量 $\vec{a},\vec{b},\vec{c}$ 满足关系式 $\vec{a}+\vec{b}+\vec{c}=0$,则 $\vec{a}\times\vec{b}=$\blankbox。
        \fourchoices{$\vec{c}\times\vec{b}$}{$\vec{b}\times\vec{c}$}{$\vec{a}\times\vec{c}$}{$\vec{b}\times\vec{a}$}
    \end{bbox}

    \begin{bbox}[choice]
        \qitem 已知 $\overrightarrow{OA}=\overrightarrow{i}+3\overrightarrow{k}$,$\overrightarrow{OB}=\overrightarrow{j}+3\overrightarrow{k}$,则 $\bigtriangleup OAB$ 的面积为\blankbox。
        \vspace{2mm} 
        \fourchoices{$19$}{$\dfrac{1}{2}\sqrt{19}$}{$\sqrt{19}$}{$29$}
    \end{bbox}

    \begin{bbox}[choice]
        \qitem 非零向量 $\vec{a},\vec{b},\vec{c}$ 共面的充分必要条件是\blankbox。
        \fourchoices
        {$\vec{a}\cdot\left( \vec{b}\times\vec{c}\right)=0$}
        {$\vec{a}\cdot\left( \vec{b}\cdot\vec{c}\right)=0$}
        {$\vec{a}\times\left( \vec{b}\times\vec{c}\right)=0$}
        {$\vec{a}\times\left( \vec{b}\cdot\vec{c}\right)=0$}
    \end{bbox}

    \begin{bbox}[big]
        \qitem 已知 $\vec{a}=\left( {1,1,-4}\right)$,$\vec{b}=\left( {2,0,-2}\right)$,求:
        \begin{subqitems}
            \subqitem $\left( \vec{a}-\vec{b}\right)\cdot\vec{a}$;
            \subqitem $\left( \vec{a}\times\vec{b}\right)\cdot\left( \vec{a}+\vec{b}\right)$。
        \end{subqitems}
    \end{bbox}

    \begin{bbox}[big]
        \qitem 已知 $\vec{a}=\left( {1,1,-4}\right)$,$\vec{b}=\left( {2,0,-2}\right)$,求 ${\operatorname{Prj}}_{\vec{a}}\vec{b}$。
    \end{bbox}

    \begin{bbox}[big]
        \qitem 已知 $\left|\vec{a}\right|=1$,$\left|\vec{b}\right|=2$,$\left( \vec{a},\vec{b}\right)=60^{\circ}$,求 $\left| \vec{a}-\dfrac{1}{3}\left( \vec{a}-\vec{b}\right)\right|$。
    \end{bbox}

\end{qitems}

\subsection{曲面及其方程}
\begin{qitems}

    \begin{bbox}[blank]
        \qitem 方程 $-\dfrac{x^{2}}{4}+\dfrac{y^{2}}{9}=1$ 表示的空间曲面是\blankline。
    \end{bbox}

    \begin{bbox}[blank]
        \qitem 方程 $\dfrac{x^{2}}{9}+\dfrac{z^{2}}{4}=1$ 表示的空间曲面是\blankline。
    \end{bbox}

    \begin{bbox}[big]
        \qitem 建立以点 $\left( {1,3,-2}\right)$ 为球心,且通过坐标原点的球面方程。
    \end{bbox}

    \begin{bbox}[big]
        \qitem 将 $xoy$ 坐标面上的双曲线 $4x^{2}-9y^{2}=36$ 分别绕 $x$ 轴及 $y$ 轴旋转一周,求所生成的旋转曲面的方程。
    \end{bbox}

\end{qitems}

\subsection{空间曲线及其方程}
\begin{qitems}

    \begin{bbox}[big]
        \qitem 求曲线 $C:\left\{  \begin{matrix} x^{2}+y^{2}-z^{2}=0 \\  x-z+1=0 \end{matrix}\right.$ 关于 $xoy$ 坐标面的投影柱面方程及此曲线在 $xoy$ 坐标面的投影方程。
    \end{bbox}

\vspace{2.5cm}
    \begin{bbox}[big]
        \qitem 求母线平行于 $y$ 轴,且通过曲线 $\left\{  \begin{matrix} 2x^{2}+y^{2}+z^{2}=16 \\  x^{2}+y^{2}-z^{2}=0 \end{matrix}\right.$ 的投影柱面方程。
    \end{bbox}

\end{qitems}

\subsection{平面及其方程}
\begin{qitems}

    \begin{bbox}[choice]
        \qitem 平面 $x-y+2z-6=0$ 和 $2x+y+z-5=0$ 的夹角是\blankbox。
        \fourchoices{$\pi$}{$\dfrac{\pi}{2}$}{$\dfrac{\pi}{3}$}{$2\pi$}
    \end{bbox}

    \begin{bbox}[choice]
        \qitem 两平面 $2x-y-z=0$ 和 $x+y+z=0$ 的位置是\blankbox。
        \fourchoices{平行}{相交不垂直}{垂直}{共面}
    \end{bbox}

    \begin{bbox}[big]
        \qitem 求过点 $A\left( {5,4,3}\right)$ 且在各坐标轴上的截距相等的平面方程。
    \end{bbox}

    \begin{bbox}[big]
        \qitem 求平行于 $xoz$ 面且经过点 $\left( {2,-5,3}\right)$ 的平面方程。
    \end{bbox}

    \begin{bbox}[big]
        \qitem 求通过 $z$ 轴和点 $\left( {-3,1,-2}\right)$ 的平面方程。
    \end{bbox}

\end{qitems}

\subsection{空间直线及其方程}
\begin{qitems}

    \begin{bbox}[big]
        \qitem 求过点 $\left( {1,0,-2}\right)$ 且与平面 $3x+4y-z+6=0$ 平行,与直线 $\dfrac{x-3}{1}=\dfrac{y+2}{4}=\dfrac{z}{1}$ 垂直的直线方程。
    \end{bbox}

    \begin{bbox}[big]
        \qitem 求过点 $\left( {3,2,-1}\right)$ 且与平面 $x-4z-3=0$ 及 $2x-y-5z-1=0$ 平行的直线方程。
    \end{bbox}

    \begin{bbox}[big]
        \qitem 求通过平面 $x+y-z-2=0$ 与 $3x+y-z-5=0$ 的交线,且过点 $\left( {1,8,2}\right)$ 的平面方程。
    \end{bbox}

    \begin{bbox}[big]
        \qitem 求点 $M\left( {1,2,-1}\right)$ 到直线 $\dfrac{x-1}{2}=\dfrac{y+1}{-1}=\dfrac{z-2}{3}$ 的距离。
    \end{bbox}

\newpage
    \begin{bbox}[big]
        \qitem 求点 $M\left( {1,2,3}\right)$ 到直线 $\left\{  \begin{matrix} x+y-z=1 \\  2x+z=3 \end{matrix}\right.$ 的距离。
    \end{bbox}

    \begin{bbox}[big]
        \qitem 求点 $N\left( {-1,2,0}\right)$ 在平面 $x+2y-z+1=0$ 上的投影。
    \end{bbox}

\begin{bbox}[big]
    \qitem 确定 $\lambda$,使直线
    \[
      \dfrac{x-1}{1}=\dfrac{y+2}{2}=\dfrac{z-1}{\lambda}
    \]
    垂直于平面 ${\pi}_{1}:3x+6y+3z+25=0$,并求该直线在平面 ${\pi}_{2}:x-y+z-2=0$ 上的投影直线的方程。
\end{bbox}


\end{qitems}

\subsection{本章综合测验}
\begin{qitems}

    \begin{bbox}[choice]
        \qitem 若非零向量 $\overrightarrow{a}$ 和 $\overrightarrow{b}$ 满足 $\left| {\overrightarrow{a}-\overrightarrow{b}}\right|=\left| \overrightarrow{a}\right|+\left| \overrightarrow{b}\right|$,则\blankbox。
        \fourchoices{方向相同}{互相垂直}{方向相反}{平行}
    \end{bbox}

    \begin{bbox}[choice]
        \qitem 方程 $y^{2}+z^{2}-24x+8=0$ 表示\blankbox。
        \fourchoices{双曲柱面}{椭圆柱面}{锥面}{旋转抛物面}
    \end{bbox}

    \begin{bbox}[choice]
        \qitem 方程 $x^{2}+y^{2}+z^{2}=49$ 表示的曲面是\blankbox。
        \fourchoices{柱面}{球面}{锥面}{旋转抛物面}
    \end{bbox}

    \begin{bbox}[choice]
        \qitem 平面 $x=2z$\blankbox。
        \fourchoices{平行 $xOz$ 坐标面}{平行 $y$ 轴}{垂直 $y$ 轴}{通过 $y$ 轴}
    \end{bbox}

    \begin{bbox}[choice]
        \qitem 曲面 $x^{2}+y^{2}+z^{2}=9$ 与 $x+y=1$ 的交线在 $xoy$ 面上的投影为\blankbox。
        \fourchoices{椭圆柱面}{椭圆曲线}{两平行平面}{线段}
    \end{bbox}

    \begin{bbox}[choice]
        \qitem 直线 $L:\dfrac{x+3}{-2}=\dfrac{y+4}{-7}=\dfrac{z}{3}$ 与平面 $\pi:4x-2y-2z=3$ 的关系是\blankbox。
        \fourchoices{平行}{垂直相交}{L 在 $\pi$ 上}{相交但不垂直}
    \end{bbox}

    \begin{bbox}[choice]
        \qitem 直线 $L:\dfrac{x}{3}=\dfrac{y}{-2}=\dfrac{z}{7}$ 和平面 $3x-2y+7z=8$ 的关系是\blankbox。
        \fourchoices{平行}{垂直相交}{L 在 $\pi$ 上}{相交但不垂直}
    \end{bbox}

    \begin{bbox}[choice]
        \qitem 设直线 $\dfrac{x}{0}=\dfrac{y}{4}=\dfrac{z}{-3}$,则该直线必定\blankbox。
        \fourchoices{过原点且垂直于 $x$ 轴}{过原点且平行于 $x$ 轴}{不过原点,但垂直于 $x$ 轴}{不过原点,且不平行于 $x$ 轴}
    \end{bbox}


    \begin{bbox}[blank]
        \qitem 向量的终点在点 $B\left( {2,-1,7}\right)$,它在坐标轴上的投影依次是 $4$、$-4$、$7$,这个向量的起点 $A$ 的坐标为\blankline。\vspace{2mm}
    \end{bbox}

    \begin{bbox}[blank]
        \qitem 将 $xoz$ 坐标面上的曲线 $z^{2}=5x$ 绕 $x$ 轴旋转所生成的旋转曲面方程为\blankline。
    \end{bbox}

    \begin{bbox}[blank]
        \qitem 过点 $\left( {2,-5,3}\right)$ 且平行于 $xoz$ 面的平面方程为\blankline。
    \end{bbox}

    \begin{bbox}[blank]
        \qitem 过点 $\left( {2,4,-1}\right)$ 且平行于 $S=\left( {1,3,4}\right)$ 的直线方程为\blankline。
    \end{bbox}

    \begin{bbox}[blank]
        \qitem 通过点 $M\left( {1,2,3}\right)$ 且与直线 $L:x=2+3t$,$y=2t$,$z=-1+t$ 垂直的平面方程为\newline\blankline。
    \end{bbox}


    \begin{bbox}[big]
        \qitem 已知 ${M}_{1}\left( {4,\sqrt{2},1}\right)$,${M}_{2}\left( {3,0,2}\right)$,求向量 ${M}_{1}{M}_{2}$ 的模、方向余弦和方向角。
    \end{bbox}

    \begin{bbox}[big]
        \qitem 设向量 $\vec{r}$ 的模是 4,它与轴 $\vec{u}$ 的夹角是 $60^{\circ}$,求 $\vec{r}$ 在轴 $\vec{u}$ 上的投影。
    \end{bbox}

    \begin{bbox}[big]
        \qitem 求向量 $\vec{b}=\vec{i}-\vec{j}+3\vec{k}$ 与 $\vec{c}=\vec{i}-2\vec{j}$ 的夹角余弦。
    \end{bbox}

    \begin{bbox}[big]
        \qitem 设 $\vec{a}=\left( {x,y,z}\right)$,$\vec{b}=\left( {2,0,5}\right)$,$\vec{c}=\left( {3,0,0}\right)$,问当 $x,y,z$ 取何值时,$\vec{a}$ 与 $\vec{b}$ 平行;取何值时 $\vec{a}$ 与 $\vec{c}$ 平行。
    \end{bbox}

\newpage
    \begin{bbox}[big]
        \qitem 已知 ${M}_{1}\left( {1,-1,2}\right)$,${M}_{2}\left( {3,3,1}\right)$,${M}_{3}\left( {3,1,3}\right)$,求与 $\overrightarrow{{M}_{1}{M}_{2}}$、$\overrightarrow{{M}_{2}{M}_{3}}$ 同时垂直的单位向量。
    \end{bbox}

    \begin{bbox}[big]
        \qitem 化直线方程 $\left\{  \begin{matrix} x-y+z+5=0 \\  5x-8y+4z+36=0 \end{matrix}\right.$ 为对称式方程和参数方程。
    \end{bbox}

    \begin{bbox}[big]
        \qitem 求直线 $L:\dfrac{x-1}{1}=\dfrac{y}{1}=\dfrac{z+1}{-1}$ 在平面 $\Pi:x-y+2z-1=0$ 上的投影直线 ${L}_{0}$ 的方程。
    \end{bbox}

    \begin{bbox}[big]
        \qitem 试证直线 $\dfrac{x-1}{1}=\dfrac{y-1}{1}=\dfrac{z+3}{-2}$ 在平面 $x+y+z+1=0$ 上。
    \end{bbox}

\end{qitems}

\section{多元函数微分法及其应用}

\subsection{多元函数的基本概念}
\begin{qitems}

    \begin{bbox}[choice]
        \qitem 函数 $z=\ln(-x-y)$ 的定义域是\blankbox。
        \fourchoices{$\{(x,y)\mid x<0,\;y<0\}$}{$\{(x,y)\mid x+y\le 0\}$}{$\{(x,y)\mid x+y<0\}$}{在 $xoy$ 平面上处处无定义}
    \end{bbox}

    \begin{bbox}[big]
        \qitem 计算 $\displaystyle \lim_{\substack{x \rightarrow 0\\ y \rightarrow 0}}\frac{x^{2}+y^{2}}{\sqrt{x^{2}+y^{2}+1}-1}$。
    \end{bbox}

    \begin{bbox}[big]
        \qitem 计算 $\displaystyle \lim_{\substack{x \rightarrow 0\\ y \rightarrow 0}}\left( x+y \right)\sin \frac{1}{x}\sin y$。
    \end{bbox}

    \begin{bbox}[big]
        \qitem 试证 $\displaystyle \lim_{\substack{x \rightarrow 0\\ y \rightarrow 0}}\frac{\ln \left( 1+xy\right)}{x+\sin y}$ 的极限不存在。
    \end{bbox}

\end{qitems}

\subsection{偏导数}
\begin{qitems}

    \begin{bbox}[choice]
        \qitem 设 $z=x+(y-2)\arcsin\sqrt{\frac{x}{y}}$,那么 $\left.\frac{\partial z}{\partial y}\right|_{(1,2)}=$\blankbox。
        \fourchoices{$0$}{$1$}{$\dfrac{\pi}{2}$}{$\dfrac{\pi}{4}$}
    \end{bbox}

    \begin{bbox}[choice]
        \qitem 设 $z=(1+x)^{x+y}$,则 $\left.\frac{\partial z}{\partial x}\right|_{(1,1)}=$\blankbox。
        \fourchoices{$1+\ln 2$}{$4\left( 1+\ln 2\right)$}{$4$}{$8$}
    \end{bbox}

    \begin{bbox}[choice]
        \qitem 函数 $f(x,y)=\begin{cases}\dfrac{2xy}{x^{2}+y^{2}}, & x^{2}+y^{2}\neq 0,\\ 0, & x^{2}+y^{2}=0,\end{cases}$ 在点 $(0,0)$ 处\blankbox。
        \fourchoices{连续且可导}{不连续且不可导}{连续但不可导}{可导但不连续}
    \end{bbox}

    \begin{bbox}[big]
        \qitem 证明:$f(x,y)=\begin{cases}\dfrac{x y^{2}}{x^{2}+y^{4}}, & (x,y)\neq (0,0),\\ 0, & (x,y)=(0,0),\end{cases}$ 在点 $(0,0)$ 处不连续,但一阶偏导数存在。
    \end{bbox}

    \begin{bbox}[big]
        \qitem 设 $u=(x^{2}+y z^{3})^{3}$,求 $\dfrac{\partial u}{\partial x},\dfrac{\partial u}{\partial y},\dfrac{\partial u}{\partial z}$。
    \end{bbox}

    \begin{bbox}[big]
        \qitem 设 $z=\arctan \dfrac{y}{x}$,求 $\dfrac{\partial^{2} z}{\partial x^{2}},\dfrac{\partial^{2} z}{\partial x\partial y}$。
    \end{bbox}

\end{qitems}

\subsection{全微分}
\begin{qitems}

    \begin{bbox}[choice]
        \qitem 函数 $f(x,y)=\begin{cases}\dfrac{x^{2}y^{2}}{x^{4}+y^{4}}, & (x,y)\neq (0,0),\\ 0, & (x,y)=(0,0),\end{cases}$ 在点 $(0,0)$ 处结论正确的是\blankbox。
        \fourchoices{连续但不可微}{可微}{可导但不可微}{既不连续又不可导}
    \end{bbox}

    \begin{bbox}[choice]
        \qitem 函数 $z=f(x,y)$ 在点 $(x_{0},y_{0})$ 处偏导数存在、连续、可微分三者的关系是\blankbox。
        \fourchoices{可微必连续}{偏导数存在必可微}{连续必可微}{偏导数存在必连续}
    \end{bbox}

    \begin{bbox}[big]
        \qitem 设 $z=e^{\frac{y}{x}}+\sin(xy)$,计算 $\mathrm{d}z$。
    \end{bbox}

    \begin{bbox}[big]
        \qitem 设 $z=\dfrac{y}{\sqrt{x^{2}+y^{2}}}$,计算 $\mathrm{d}z$。
    \end{bbox}

    \begin{bbox}[big]
        \qitem 设 $z=x^{y}+y^{x}$,求 $\mathrm{d}z$。
    \end{bbox}

\end{qitems}

\subsection{多元复合函数的求导法则}
\begin{qitems}

    \begin{bbox}[big]
        \qitem 设 $z=f(\sin x,e^{x+2y})$,其中 $f$ 具有一阶连续偏导数,求 $\mathrm{d}z$。
    \end{bbox}

    \begin{bbox}[big]
        \qitem 设 $z=f(2x-y)+g(x,xy)$,其中 $f(t)$ 二阶可导,$g(u,v)$ 具有二阶连续偏导数,求 $\dfrac{\partial z}{\partial x}$。
    \end{bbox}

    \begin{bbox}[big]
        \qitem 设 $z=f(\sin x,\cos y,e^{x+y})$,其中 $f$ 具有二阶连续偏导数,求 $\dfrac{\partial z}{\partial x},\dfrac{\partial^{2} z}{\partial x\partial y}$。
    \end{bbox}

\end{qitems}

\subsection{隐函数的求导公式}
\begin{qitems}

    \begin{bbox}[big]
        \qitem 设方程 $e^{z}=xyz$ 确定函数 $z=z(x,y)$,利用两种方法计算 $\dfrac{\partial z}{\partial x},\dfrac{\partial z}{\partial y}$。
    \end{bbox}

    \begin{bbox}[big]
        \qitem 设 $z=z(x,y)$ 由 $G(xyz,x+y+z)=0$ 所确定,其中 $G$ 具有连续的一阶偏导数,求 $\dfrac{\partial z}{\partial x},\dfrac{\partial z}{\partial y}$。
    \end{bbox}

    \begin{bbox}[big]
        \qitem 设 $x,y,z,u,v$ 满足方程 $\left\{\begin{array}{l} x^{2}u+yz=v,\\ \sin x+2zv=u,\end{array}\right.$,取 $x,y,z$ 为自变量,求 $\dfrac{\partial u}{\partial x}$。
    \end{bbox}

\end{qitems}

\subsection{多元函数微分学的几何应用}
\begin{qitems}

    \begin{bbox}[choice]
        \qitem 曲线 $x=t,y=t^{2},z=t^{3}$ 在点 $(1,1,1)$ 处的法平面方程是\blankbox。
        \fourchoices{$2x+3y-z=6$}{$x+2y+3z=6$}{$x+y-z=2$}{$x-2y+3z=3$}
    \end{bbox}

    \begin{bbox}[choice]
        \qitem 若曲线 $x=\cos t,y=2\sin t,z=t^{2}$ 在 $t=\dfrac{\pi}{2}$ 点处的一个切向量与 $oz$ 轴正方向成钝角,则此向量与 $yz$ 平面夹角的正弦值为\blankbox。
        \fourchoices{$\dfrac{1}{\sqrt{1+\pi^{2}}}$}{$-\dfrac{1}{\sqrt{1+\pi^{2}}}$}{$\dfrac{\pi}{\sqrt{1+\pi^{2}}}$}{$-\dfrac{\pi}{\sqrt{1+\pi^{2}}}$}
    \end{bbox}

    \begin{bbox}[big]
        \qitem 求曲线 $\Gamma:\left\{\begin{array}{l} x^{2}+y^{2}+z^{2}=6,\\ x^{2}+y^{2}-z^{2}=4,\end{array}\right.$ 在点 $(2,1,1)$ 处的法平面方程。
    \end{bbox}

    \begin{bbox}[big]
        \qitem 求曲面 $x^{3}+3xy^{2}+z^{3}+2x^{2}z+yz^{2}-35z-59=0$ 在点 $(2,-1,-3)$ 处的切平面方程。
    \end{bbox}

    \begin{bbox}[big]
        \qitem 求椭球面 $x^{2}+2y^{2}+z^{2}=1$ 上平行于平面 $x-y+2z=0$ 的切平面方程。
    \end{bbox}

\end{qitems}

\subsection{方向导数与梯度}
\begin{qitems}

    \begin{bbox}[big]
        \qitem 求函数 $u=xyz$ 在点 $(5,1,2)$ 处沿从点 $(5,1,2)$ 到点 $(9,4,14)$ 方向的方向导数。
    \end{bbox}

    \begin{bbox}[blank]
        \qitem 函数 $u=x^{2}+y^{2}+z^{2}$ 沿曲线 $x=2t,y=t^{3},z=\dfrac{6}{\pi}\sin(\pi t)$ 在点 $(2,1,0)$ 处的切线方向的方向导数是\blankline(切线方向沿 $t$ 增加的方向)。
    \end{bbox}

    \begin{bbox}[blank]
        \qitem 函数 $u=\ln\left( x^{2}+y^{2}+z^{2}\right)$ 在点 $M(1,2,-2)$ 处的梯度为\blankline。
    \end{bbox}

    \begin{bbox}[blank]
        \qitem 求函数 $u=xyz-2yz-3$ 在点 $(1,3,1)$ 处的最大方向导数\blankline。
    \end{bbox}

\end{qitems}

\subsection{多元函数的极值及其求法}
\begin{qitems}

    \begin{bbox}[choice]
        \qitem 函数 $f(x,y)=2x^{2}+ax+xy^{2}+2y$ 在点 $(1,-1)$ 处取得极值,则常数 $a=$\blankbox。
        \fourchoices{$5$}{$-5$}{$0$}{无法确定}
    \end{bbox}

    \begin{bbox}[big]
        \qitem 求函数 $z=x^{2}-3xy+3y^{2}-2x+3y+4$ 的极值。
    \end{bbox}

    \begin{bbox}[big]
        \qitem 利用拉格朗日乘数法,求函数 $u=x-2y+3z$ 在条件 $x^{2}+2y^{2}+3z^{2}=6$ 下的极大值或极小值。
    \end{bbox}

    \begin{bbox}[big]
        \qitem 求函数 $z=xy$ 在闭区域 $x\ge 0,y\ge 0,x+y\le 1$ 上的最大值。
    \end{bbox}

    \begin{bbox}[big]
        \qitem 在曲面 $\Sigma:2x^{2}+y^{2}-z^{2}-2xy+1=0$ 上求一点,使其到原点的距离最小。
    \end{bbox}

\newpage

    \begin{bbox}[big]
        \qitem 在 $xoy$ 平面上求一点,使它到 $x=0,y=0$ 及 $x+2y-16=0$ 三平面的距离的平方和为最小。
    \end{bbox}

    \begin{bbox}[big]
        \qitem 求内接于半径为 $8$ 的球且有最大体积的长方体。
    \end{bbox}

\end{qitems}

\subsection{本章综合测验}

\textbf{一、选择题}
\begin{qitems}

    \begin{bbox}[choice]
        \qitem 函数 $f(x,y)=\begin{cases}\dfrac{xy}{x^{2}+y^{2}}, & (x,y)\neq (0,0),\\ 0, & (x,y)=(0,0),\end{cases}$ 在点 $(0,0)$ 处结论正确的是\blankbox。
        \fourchoices{一阶偏导数存在}{连续}{可微}{二重极限存在}
    \end{bbox}

    \begin{bbox}[choice]
        \qitem 函数 $z=f(x,y)$ 在点 $(x_{0},y_{0})$ 处可微分是它在该点偏导数存在的\blankbox。
        \fourchoices{必要而非充分条件}{充分而非必要条件}{充分必要条件}{既非充分又非必要条件}
    \end{bbox}

    \begin{bbox}[choice]
        \qitem 设 $z=x+(y-1)\arcsin\sqrt{\frac{x}{y}}$,则 $\left.\frac{\partial z}{\partial x}\right|_{(0,1)}=$\blankbox。
        \fourchoices{$-\dfrac{1}{4}$}{$\dfrac{1}{4}$}{$-\dfrac{1}{2}$}{$1$}
    \end{bbox}

    \begin{bbox}[choice]
        \qitem 曲面 $xyz=8$ 上平行于平面 $x+y+z+3=0$ 的切平面方程是\blankbox。
        \fourchoices{$x+y+z=3$}{$x+y+z=1$}{$x+y+z=6$}{$x+y+z=0$}
    \end{bbox}

    \begin{bbox}[choice]
        \qitem 设函数 $f(x,y)$ 在点 $(0,0)$ 的某邻域内有定义,且 $f_{x}(0,0)=3,f_{y}(0,0)=-1$,则\blankbox。
        \fourchoices{$\left. \mathrm{d}z\right|_{(0,0)}=3\mathrm{d}x-\mathrm{d}y$}{曲面 $z=f(x,y)$ 在点 $(0,0,f(0,0))$ 的一个法向量为 $(3,-1,1)$}{曲线 $\left\{\begin{array}{l} z=f(x,y)\\ y=0 \end{array}\right.$ 在点 $(0,0,f(0,0))$ 的一个切向量为 $(1,0,3)$}{曲线 $\left\{\begin{array}{l} z=f(x,y)\\ y=0 \end{array}\right.$ 在点 $(0,0,f(0,0))$ 的一个切向量为 $(3,0,1)$}
    \end{bbox}

    \begin{bbox}[choice]
        \qitem 曲线 $x=2t,y=t^{3},z=\dfrac{6}{\pi}\sin(\pi t)$ 在点 $(2,1,0)$ 处的法平面方程为\blankbox。
        \fourchoices{$2x+y-7=0$}{$2x+3y-6z-7=0$}{$2x-3y+6z-7=0$}{$2x+y+z+7=0$}
    \end{bbox}

    \begin{bbox}[choice]
        \qitem 曲面 $3x^{2}+2y^{2}+z^{2}=20$ 在点 $(-1,-2,3)$ 处的切平面与 $xoy$ 面的夹角的余弦为\blankbox。
        \fourchoices{$\dfrac{3}{\sqrt{34}}$}{$-\dfrac{3}{\sqrt{34}}$}{$-\sqrt{34}$}{$\sqrt{34}$}
    \end{bbox}

    \begin{bbox}[choice]
        \qitem 设函数 $z=\ln \sqrt{x^{2}+y^{2}}$,则 $\dfrac{\partial^{2} z}{\partial x^{2}}+\dfrac{\partial^{2} z}{\partial y^{2}}=$\blankbox。
        \fourchoices{$0$}{$-1$}{$1$}{$2$}
    \end{bbox}

\end{qitems}
\vspace{3mm}
\textbf{二、填空题}
\vspace{3mm}
\begin{qitems}

    \begin{bbox}[blank]
        \qitem 设 $u=\dfrac{\cos x^{2}}{y}+\left( xy\right)^{\frac{y}{x}}$,则 $\left.\dfrac{\partial u}{\partial x}\right|_{(x,1)}=$\blankline。
    \end{bbox}

    \begin{bbox}[blank]
        \qitem 设 $z=\ln\left( x^{2}+y^{2}+1\right)+e^{xy}$,则 $\dfrac{\partial z}{\partial x}=$\blankline。
    \end{bbox}

    \begin{bbox}[blank]
        \qitem 设 $f(x,y)=\begin{cases}\dfrac{xy}{\sqrt{x^{2}+y^{2}}}, & x^{2}+y^{2}\neq 0,\\ 0, & x^{2}+y^{2}=0,\end{cases}$ 则 $\left.\dfrac{\partial f}{\partial x}\right|_{(0,0)}=$\blankline。
    \end{bbox}

    \begin{bbox}[blank]
        \qitem 函数 $u=xyz-2yz-3$ 在点 $(1,1,1)$ 处的梯度为\blankline。
    \end{bbox}

    \begin{bbox}[blank]
        \qitem 函数 $z=x^{2}+y^{2}$ 在点 $(1,2)$ 处沿从点 $(1,2)$ 到点 $(2,2+\sqrt{3})$ 方向的方向导数为\blankline。
    \end{bbox}

\end{qitems}
\vspace{6mm}
\textbf{三、计算题}
\vspace{3mm}
\begin{qitems}

    \begin{bbox}[big]
        \qitem 设 $z=z(x,y)$ 由方程 $x e^{z}-3yz=5$ 所确定,求 $\mathrm{d}z$。
    \end{bbox}

    \begin{bbox}[big]
        \qitem 已知方程 $\dfrac{x}{z}=\ln \dfrac{z}{y}$ 确定了函数 $z=f(x,y)$,求 $\dfrac{\partial z}{\partial x}$。
    \end{bbox}

    \newpage
    \begin{bbox}[big]
        \qitem 设函数 $z=z(x,y)$ 由方程 $z=\varphi(x-y,y-z)$ 所确定,其中 $\varphi(u,v)$ 有一阶连续偏导数,求 $\dfrac{\partial z}{\partial x},\dfrac{\partial z}{\partial y}$。
    \end{bbox}

    \begin{bbox}[big]
        \qitem 求表面积为 $9$ 而体积为最大的长方体的体积。
    \end{bbox}

    \begin{bbox}[big]
        \qitem 利用拉格朗日乘数法,求函数 $u=x^{2}+y^{2}+z^{2}$ 在条件 $x+2y+2z=18,\;x>0,y>0,z>0$ 下的最小值。
    \end{bbox}

\end{qitems}

\section{重积分及其应用}

\subsection{二重积分的概念与性质}
\begin{qitems}

    \begin{bbox}[blank]
        \qitem 判断:二重积分 $\iint_{D} f(x,y)\,\mathrm{d}x\mathrm{d}y$ 的几何意义是以 $z=f(x,y)$ 为曲顶、以 $D$ 为底的曲顶柱体的体积\blankbox。
    \end{bbox}

    \begin{bbox}[big]
        \qitem 由二重积分的几何意义,求 $\iint_{D}\sqrt{a^{2}-x^{2}-y^{2}}\,\mathrm{d}\sigma$,其中 $D$ 为 $xoy$ 平面上的区域 $x^{2}+y^{2}\le a^{2},\;x\ge 0,\;y\ge 0$。
    \end{bbox}

    \begin{bbox}[choice]
        \qitem 估计积分 $I=\iint_{|x|+|y|\le 10}\dfrac{1}{100+\cos^{2}x+\sin^{2}y}\,\mathrm{d}x\mathrm{d}y$ 的值,正确的是\blankbox。
        \fourchoices{$\dfrac{1}{2}\le I\le 1.04$}{$1.04\le I\le 1.96$}{$1.96\le I\le 2$}{$2\le I\le 2.14$}
    \end{bbox}

    \begin{bbox}[big]
        \qitem 设 $D$ 由 $x=0,y=0,x+y=\dfrac{1}{2},x+y=1$ 围成,确定以下积分大小的顺序:\newline
        \vspace{-9mm}
        \begin{flalign*}
&         I_{1}=\iint_{D}\left[\ln(x+y)\right]^{7}\,\mathrm{d}x\,\mathrm{d}y,\quad
          I_{2}=\iint_{D}(x+y)^{7}\,\mathrm{d}x\,\mathrm{d}y,\quad
          I_{3}=\iint_{D}\left[\sin(x+y)\right]^{7}\,\mathrm{d}x\,\mathrm{d}y &&
        \end{flalign*}
    \end{bbox}

    \begin{bbox}[big]
        \qitem 设 $f(x,y)$ 是有界闭区域 $D: x^{2}+y^{2}\le a^{2}$ 上的连续函数,计算 $\displaystyle \lim_{a\to 0}\frac{1}{\pi a^{2}}\iint_{D}f(x,y)\,\mathrm{d}x\mathrm{d}y$。
    \end{bbox}

\end{qitems}

\subsection{二重积分的计算方法}
\begin{qitems}

    \begin{bbox}[choice]
        \qitem 设 $f(x,y)$ 是连续函数,$a>0$,则 $\displaystyle \int_{0}^{a}\mathrm{d}x\int_{0}^{x}f(x,y)\,\mathrm{d}y=$\blankbox。
        \vspace{1.5mm}

        {\renewcommand{\arraystretch}{1.8}
        \fourchoices
          {$\displaystyle \int_{0}^{a}\mathrm{d}y\int_{y}^{a}f(x,y)\,\mathrm{d}x$}
          {$\displaystyle \int_{0}^{a}\mathrm{d}y\int_{0}^{a}f(x,y)\,\mathrm{d}x$}
          {$\displaystyle \int_{0}^{a}\mathrm{d}y\int_{0}^{y}f(x,y)\,\mathrm{d}x$}
          {$\displaystyle \int_{0}^{a}\mathrm{d}y\int_{a}^{y}f(x,y)\,\mathrm{d}x$}
        }
    \end{bbox}
\vspace{3mm}
\begin{bbox}[choice]
    \qitem 二次积分 $I=\displaystyle \int_{0}^{1}\mathrm{d}y\int_{0}^{\sqrt{1-y}}f(x,y)\,\mathrm{d}x$ 换序正确的结果是\blankbox。
    \vspace{1.5mm}

    {\renewcommand{\arraystretch}{1.8}
    \fourchoices
      {$\displaystyle \int_{0}^{1}\mathrm{d}x\int_{0}^{1-x^{2}}f(x,y)\,\mathrm{d}y$}
      {$\displaystyle \int_{0}^{1}\mathrm{d}x\int_{0}^{\sqrt{1-x}}f(x,y)\,\mathrm{d}y$}
      {$\displaystyle \int_{0}^{1}\mathrm{d}x\int_{1-x^{2}}^{1}f(x,y)\,\mathrm{d}y$}
      {$\displaystyle \int_{0}^{1}\mathrm{d}x\int_{\sqrt{1-x}}^{1}f(x,y)\,\mathrm{d}y$}
    }
\end{bbox}
\vspace{3mm}

    \begin{bbox}[choice]
        \qitem 设域 $D: x^{2}+y^{2}\le 4$,$f$ 为域 $D$ 上的连续函数,则 $\iint_{D}f\left( \sqrt{x^{2}+y^{2}}\right)\,\mathrm{d}x\mathrm{d}y=$\blankbox。
        \vspace{1.5mm}

        {\renewcommand{\arraystretch}{1.8}
        \fourchoices
          {$2\pi \displaystyle \int_{0}^{2}\rho f(\rho)\,\mathrm{d}\rho$}
          {$4\pi \displaystyle \int_{0}^{2}\rho f(\rho)\,\mathrm{d}\rho$}
          {$2\pi \displaystyle \int_{0}^{2}f(\rho^{2})\,\mathrm{d}\rho$}
          {$4\pi \displaystyle \int_{0}^{r}\rho f(\rho)\,\mathrm{d}\rho$}
        }
    \end{bbox}
\vspace{3mm}
    \begin{bbox}[choice]
        \qitem 若区域 $D:(x-1)^{2}+y^{2}\le 1$,则 $\iint_{D}f(x,y)\,\mathrm{d}x\mathrm{d}y$ 化为极坐标累次积分为\blankbox。
        \vspace{1.5mm}

        {\renewcommand{\arraystretch}{1.8}
        \fourchoices
          {$\displaystyle \int_{0}^{\pi}\mathrm{d}\theta\int_{0}^{2\cos\theta}F(\rho,\theta)\,\mathrm{d}\rho$}
          {$\displaystyle \int_{-\pi}^{\pi}\mathrm{d}\theta\int_{0}^{2\cos\theta}F(\rho,\theta)\,\mathrm{d}\rho$}
          {$\displaystyle \int_{-\frac{\pi}{2}}^{\frac{\pi}{2}}\mathrm{d}\theta\int_{0}^{2\cos\theta}F(\rho,\theta)\,\mathrm{d}\rho$}
          {$\displaystyle 2\int_{0}^{\frac{\pi}{2}}\mathrm{d}\theta\int_{0}^{2\cos\theta}F(\rho,\theta)\,\mathrm{d}\rho$}
        }
    \end{bbox}
\vspace{3mm}
    \begin{bbox}[blank]
        \qitem 设 $D$ 由 $x=y,\;y^{2}=4x$ 所围成,将 $\displaystyle \iint_{D}f(x,y)\,\mathrm{d}x\mathrm{d}y$ 化为先 $y$ 后 $x$ 的二次积分为\blankline。
    \end{bbox}

    \begin{bbox}[blank]
        \qitem 设 $D$ 由 $x=y,\;y=2$ 及 $y=\dfrac{1}{x}(x>0)$ 所围成,将 $\displaystyle\iint_{D}f(x,y)\,\mathrm{d}x\mathrm{d}y$ 化为先 $x$ 后 $y$ 的二次积分为\blankline。
    \end{bbox}

    \begin{bbox}[blank]
        \qitem 设 $f(x,y)$ 为连续函数,交换二次积分的次序:$\displaystyle \int_{-1}^{0}\mathrm{d}y\int_{2}^{1-y}f(x,y)\,\mathrm{d}x=$\blankline。
    \end{bbox}

\newpage
    \begin{bbox}[big]
        \qitem 计算 $\displaystyle I=\iint_{x^{2}+y^{2}\le 4}\left( 1+\sqrt[3]{xy}\right)\,\mathrm{d}\sigma$。
    \end{bbox}

    \begin{bbox}[big]
        \qitem 计算 $\displaystyle \iint_{D}\dfrac{x^{2}}{y^{2}}\,\mathrm{d}x\mathrm{d}y$,其中 $D$ 是由 $y=x,xy=1,y=2$ 围成的区域。
    \end{bbox}

    \begin{bbox}[big]
        \qitem 交换积分次序 $\displaystyle \int_{0}^{1}\mathrm{d}y\int_{1-y}^{1+y^{2}}f(x,y)\,\mathrm{d}x$。
    \end{bbox}

    \begin{bbox}[big]
        \qitem 将 $\iint_{D}f(x,y)\,\mathrm{d}x\mathrm{d}y$ 化为极坐标下的二次积分,其中 $D:\left\{\begin{array}{l} 0\le x\le 1,\\ 0\le y\le 1.\end{array}\right.$
    \end{bbox}

    \begin{bbox}[big]
        \qitem 计算 $\displaystyle \iint_{D}\sqrt{R^{2}-x^{2}-y^{2}}\,\mathrm{d}x\mathrm{d}y$,其中 $D: x^{2}+y^{2}\le R y$。
    \end{bbox}

    \begin{bbox}[big]
        \qitem 计算 $\displaystyle \iint_{D}e^{\max\{x^{2},y^{2}\}}\,\mathrm{d}\sigma$,其中 $D=\{(x,y)\mid 0\le x\le 1,0\le y\le 1\}$。
    \end{bbox}

    \begin{bbox}[big]
        \qitem 计算 $\displaystyle \int_{0}^{\pi}\mathrm{d}x\int_{x}^{\pi}\frac{\sin y}{y}\,\mathrm{d}y$。
    \end{bbox}

    \begin{bbox}[big]
        \qitem 计算 $\displaystyle \int_{0}^{1}f(x)\,\mathrm{d}x$,其中 $f(x)=\displaystyle \int_{1}^{x}e^{-t^{2}}\,\mathrm{d}t$。
    \end{bbox}

    \begin{bbox}[big]
        \qitem 设 $f(x,y)$ 连续,且 $\displaystyle f(x,y)=xy+\iint_{D}f(u,v)\,\mathrm{d}u\mathrm{d}v$,其中 $D$ 是由 $y=0,y=x^{2},x=1$ 所围成的区域,求 $f(x,y)$。
    \end{bbox}

\end{qitems}

\subsection{三重积分}
\begin{qitems}

    \begin{bbox}[big]
        \qitem 对于 $\displaystyle \iiint_{\Omega}f(x,y,z)\,\mathrm{d}v$,其中 $\Omega$ 是 $z=\sqrt{4-x^{2}-y^{2}}$ 与 $z=\sqrt{3(x^{2}+y^{2})}$ 围成的区域:\vspace{3mm}\newline
        (1) 化为直角坐标系、柱面坐标系及球面坐标系下的三次积分;\newline
        (2) 若 $f(x,y,z)=\sqrt{x^{2}+y^{2}+z^{2}}$;\newline
        选用一种积分次序计算此积分。
    \end{bbox}

    \begin{bbox}[big]
        \qitem 用“先二后一”法与柱坐标计算 $\displaystyle I=\iiint_{\Omega}e^{|z|}\,\mathrm{d}v$,其中 $\Omega$ 为 $x^{2}+y^{2}+z^{2}\le 1$。
    \end{bbox}

    \begin{bbox}[big]
    \begingroup
    \linespread{2}\selectfont
        \qitem 计算 $\displaystyle \iiint_{\Omega}(y^{2}+z^{2})\,\mathrm{d}v$,其中 $\Omega$ 为 $xoy$ 平面上曲线 $y=2x$ 绕 $x$ 轴旋转而成的曲面与平面 $x=5$ 所围成的闭区域。
    \par
    \endgroup
    \end{bbox}


    \begin{bbox}[big]
        \qitem 在球坐标下计算 $\displaystyle I=\iiint_{\Omega}z\,\mathrm{d}v$,其中 $\Omega$ 是由 $z=\sqrt{x^{2}+y^{2}}$,$z=\sqrt{1-x^{2}-y^{2}}$ 所围。
    \end{bbox}
\newpage
    \begin{bbox}[big]
        \qitem 设 $f(u)$ 是可微函数,$\displaystyle F(t)=\iiint_{x^{2}+y^{2}+z^{2}\le t^{2}}f(x^{2}+y^{2}+z^{2})\,\mathrm{d}x\mathrm{d}y\mathrm{d}z$,求 $F'(t)$。
    \end{bbox}

\end{qitems}

\subsection{重积分的应用}
\begin{qitems}

    \begin{bbox}[big]
        \qitem 求由 $z\le 6-x^{2}-y^{2},\;z\ge \sqrt{x^{2}+y^{2}},\;x^{2}+y^{2}\le 1$ 所围成的立体的体积。
    \end{bbox}

    \begin{bbox}[big]
        \qitem 求由曲线 $\left\{\begin{matrix} x^{2}-2z=0,\\ y=0 \end{matrix}\right.$ 绕 $z$ 轴旋转而成的曲面围成的几何体介于平面 $z=2,\;z=8$ 之间\vspace{4mm}\newline
        部分的体积。
    \end{bbox}

    \begin{bbox}[big]
        \qitem 设有一物体,占有空间 $\Omega: 0\le x\le 1,0\le y\le 1,0\le z\le 1$,密度为 $\mu(x,y,z)=x+y+z$,求该物体的质量。
    \end{bbox}

    \begin{bbox}[big]
        \qitem 已知均匀薄片 $D$(面密度为常数 $\rho$)由抛物线 $y=x^{2}$ 及直线 $y=1$ 所围成,求:
        \begin{subqitems}
            \subqitem 薄片的形心坐标;
            \subqitem 薄片对于直线 $y=-1$ 的转动惯量。
        \end{subqitems}
    \end{bbox}

\end{qitems}

\subsection{本章综合测验}
\vspace{3mm}
\textbf{一、选择题}
\vspace{3mm}
\begin{qitems}

    \begin{bbox}[choice]
        \qitem 比较 $\displaystyle I_{1}=\iint_{D}(x+y)^{2}\,\mathrm{d}\sigma$ 与 $\displaystyle I_{2}=\iint_{D}(x+y)^{3}\,\mathrm{d}\sigma$ 的大小,其中 $D:(x-2)^{2}+(y-1)^{2}=1$,则\blankbox。
        \fourchoices{$I_{1}=I_{2}$}{$I_{1}>I_{2}$}{$I_{1}<I_{2}$}{无法比较}
    \end{bbox}

    \begin{bbox}[choice]
        \qitem 设 $D: x^{2}+y^{2}\le a^{2}$,则 $\displaystyle \iint_{D}|xy|\,\mathrm{d}x\mathrm{d}y=$\blankbox。
        \fourchoices{$\dfrac{a^{4}}{3}$}{$\dfrac{a^{4}}{2}$}{$a^{4}$}{}
    \end{bbox}

    \begin{bbox}[choice]
        \qitem 设 $f(x,y)$ 是连续函数,交换二次积分 $\displaystyle \int_{1}^{e}\mathrm{d}x\int_{0}^{\ln x}f(x,y)\,\mathrm{d}y$ 的积分次序得\blankbox。
        \vspace{1.5mm}

        {\renewcommand{\arraystretch}{1.8}
        \fourchoices
          {$\displaystyle \int_{1}^{e}\mathrm{d}y\int_{0}^{\ln x}f(x,y)\,\mathrm{d}x$}
          {$\displaystyle \int_{e^{y}}^{e}\mathrm{d}y\int_{0}^{1}f(x,y)\,\mathrm{d}x$}
          {$\displaystyle \int_{0}^{\ln x}\mathrm{d}y\int_{1}^{e}f(x,y)\,\mathrm{d}x$}
          {$\displaystyle \int_{0}^{1}\mathrm{d}y\int_{e^{y}}^{e}f(x,y)\,\mathrm{d}x$}
        }
    \end{bbox}

    \begin{bbox}[choice]
        \qitem 设 $f(x,y)$ 是连续函数,交换二次积分 $\displaystyle \int_{1}^{2}\mathrm{d}x\int_{2-x}^{x^{2}}f(x,y)\,\mathrm{d}y$ 的积分次序得\blankbox。
        \vspace{1.5mm}

        {\renewcommand{\arraystretch}{1.9}
        \fourchoices
          {$\displaystyle \int_{0}^{1}\mathrm{d}y\int_{2-y}^{2}f(x,y)\,\mathrm{d}x$}
          {$\displaystyle \int_{0}^{1}\mathrm{d}y\int_{2-y}^{2}f(x,y)\,\mathrm{d}x+\int_{1}^{4}\mathrm{d}y\int_{\sqrt{y}}^{2}f(x,y)\,\mathrm{d}x$}
          {$\displaystyle \int_{0}^{4}\mathrm{d}y\int_{2-y}^{5y}f(x,y)\,\mathrm{d}x$}
          {$\displaystyle \int_{0}^{1}\mathrm{d}y\int_{2}^{2-y}f(x,y)\,\mathrm{d}x+\int_{1}^{4}\mathrm{d}y\int_{2}^{5y}f(x,y)\,\mathrm{d}x$}
        }
    \end{bbox}
\vspace{3mm}
    \begin{bbox}[choice]
        \qitem 设 $\Omega$ 是由 $z=\sqrt{x^{2}+y^{2}}$ 和 $z=1$ 围成的立体,则 $\displaystyle \iiint_{\Omega}\mathrm{d}x\mathrm{d}y\mathrm{d}z$ 等于\blankbox。
        \vspace{1.5mm}

        {\renewcommand{\arraystretch}{1.8}
        \fourchoices
          {$\displaystyle \int_{-1}^{1}\mathrm{d}x\int_{0}^{\sqrt{1-x^{2}}}\mathrm{d}y\int_{\sqrt{x^{2}+y^{2}}}^{1}\mathrm{d}z$}
          {$\displaystyle \int_{0}^{2\pi}\mathrm{d}\theta\int_{0}^{1}\mathrm{d}\rho\int_{\rho}^{1}\rho\,\mathrm{d}z$}
          {$\displaystyle \int_{0}^{2\pi}\mathrm{d}\theta\int_{0}^{1}\mathrm{d}\rho\int_{0}^{1}\rho\,\mathrm{d}z$}
          {$\displaystyle \int_{0}^{2\pi}\mathrm{d}\theta\int_{0}^{1}\mathrm{d}\rho\int_{0}^{\rho}\rho\,\mathrm{d}z$}
        }
    \end{bbox}
\vspace{3mm}
    \begin{bbox}[choice]
        \qitem 设 $\displaystyle I=\iiint_{\Omega}(x^{2}+y^{2}+z^{2})\,\mathrm{d}v$,$\Omega: x^{2}+y^{2}+z^{2}=1$ 球面内部,则\blankbox。
        \vspace{1.5mm}

        {\renewcommand{\arraystretch}{1.8}
        \fourchoices
          {$I$ 等于 $\Omega$ 的体积}
          {$\displaystyle \int_{0}^{2\pi}\mathrm{d}\varphi\int_{0}^{2\pi}\mathrm{d}\theta\int_{0}^{1}r^{4}\sin\theta\,\mathrm{d}r$}
          {$\displaystyle \int_{0}^{2\pi}\mathrm{d}\theta\int_{0}^{\pi}\mathrm{d}\varphi\int_{0}^{1}r^{4}\sin\varphi\,\mathrm{d}r$}
          {$\displaystyle \int_{0}^{\pi}\mathrm{d}\varphi\int_{0}^{2\pi}\mathrm{d}\theta\int_{0}^{1}r^{4}\sin\varphi\,\mathrm{d}r$}
        }
    \end{bbox}
\vspace{3mm}
    \begin{bbox}[choice]
        \qitem 球体 $x^{2}+y^{2}+z^{2}\le 4a^{2}$ 与柱体 $x^{2}+y^{2}\le 2ax$ 的公共部分体积 $V=$\blankbox。
        \vspace{1.5mm}

        {\renewcommand{\arraystretch}{1.8}
        \fourchoices
          {$\displaystyle 4\int_{0}^{\frac{\pi}{2}}\mathrm{d}\theta\int_{0}^{2a\cos\theta}\sqrt{4a^{2}-\rho^{2}}\,\mathrm{d}\rho$}
          {$\displaystyle 8\int_{0}^{\frac{\pi}{2}}\mathrm{d}\theta\int_{0}^{2a\cos\theta}\sqrt{4a^{2}-\rho^{2}}\;\rho\,\mathrm{d}\rho$}
          {$\displaystyle 4\int_{0}^{\frac{\pi}{2}}\mathrm{d}\theta\int_{0}^{2a\cos\theta}\sqrt{4a^{2}-\rho^{2}}\;\rho\,\mathrm{d}\rho$}
          {$\displaystyle 8\int_{-\frac{\pi}{2}}^{\frac{\pi}{2}}\mathrm{d}\theta\int_{0}^{2a\cos\theta}\sqrt{4a^{2}-\rho^{2}}\;\rho\,\mathrm{d}\rho$}
        }
    \end{bbox}
\vspace{3mm}
    \begin{bbox}[choice]
        \qitem 由 $x^{2}+y^{2}+z^{2}\le 2z,\;z\le x^{2}+y^{2}$ 所确定的立体体积是\blankbox。
        \vspace{1.5mm}

        {\renewcommand{\arraystretch}{1.8}
        \fourchoices
          {$\displaystyle \int_{0}^{2\pi}\mathrm{d}\theta\int_{0}^{1}\rho\,\mathrm{d}\rho\int_{\rho^{2}}^{\sqrt{1-\rho^{2}}}\mathrm{d}z$}
          {$\displaystyle \int_{0}^{2\pi}\mathrm{d}\theta\int_{0}^{\rho}\rho\,\mathrm{d}\rho\int_{1}^{1-\sqrt{1-\rho^{2}}}\mathrm{d}z$}
          {$\displaystyle \int_{0}^{2\pi}\mathrm{d}\theta\int_{0}^{1}\rho\,\mathrm{d}\rho\int_{1-\sqrt{1-\rho^{2}}}^{\rho^{2}}\mathrm{d}z$}
          {$\displaystyle \int_{0}^{2\pi}\mathrm{d}\theta\int_{0}^{1}\rho\,\mathrm{d}\rho\int_{\rho^{2}}^{1-\rho^{2}}\mathrm{d}z$}
        }
    \end{bbox}
\vspace{3mm}
\end{qitems}

\textbf{二、填空题}
\begin{qitems}

    \begin{bbox}[blank]
        \qitem $D$ 由 $x=y,y=2$ 及 $y=\dfrac{1}{x}(x>0)$ 所围成,将 $\displaystyle \iint_{D}f(x,y)\,\mathrm{d}x\mathrm{d}y$ 化为先 $y$ 后 $x$ 的二次积分为\blankline。
    \end{bbox}

    \begin{bbox}[blank]
        \qitem 积分 $\displaystyle \int_{0}^{2}\mathrm{d}x\int_{x}^{2}e^{-y^{2}}\,\mathrm{d}y$ 的值等于\blankline。
    \end{bbox}

    \begin{bbox}[blank]
        \qitem 设 $f(x)$ 为连续函数,$F(t)=\displaystyle \int_{1}^{t}\mathrm{d}y\int_{y}^{t}f(x)\,\mathrm{d}x$,则 $F'(2)=$\blankline。
    \end{bbox}

    \begin{bbox}[blank]
        \qitem $\Omega: 0\le x\le 1,0\le y\le 1,-1\le z\le 1$,则 $\displaystyle \iiint_{\Omega}z\ln(x^{2}+y^{2}+z^{2})\,\mathrm{d}v=$\blankline。
    \end{bbox}

    \begin{bbox}[blank]
        \qitem 设 $\Omega$ 为单位球体 $x^{2}+y^{2}+z^{2}\le 1$,计算 $\displaystyle \iiint_{\Omega}x^{2}\,\mathrm{d}v=$\blankline。
    \end{bbox}

\end{qitems}

\newpage
\textbf{三、计算题}
\vspace{3mm}
\begin{qitems}

    \begin{bbox}[big]
        \qitem 计算 $\displaystyle I=\iint_{D}\dfrac{x}{y^{2}}\,\mathrm{d}x\mathrm{d}y$,其中 $D$ 由 $x=2,y=x,xy=1$ 所围成。
    \end{bbox}

    \begin{bbox}[big]
        \qitem 计算 $\displaystyle I=\iint_{D}|xy|\,\mathrm{d}\sigma$,其中 $D: x^{2}+y^{2}\le a^{2}$。
    \end{bbox}

    \begin{bbox}[big]
        \qitem 计算 $\displaystyle I=\iiint_{\Omega}\mathrm{d}v$,其中 $\Omega$ 由 $0\le z\le x^{2}+y^{2},\;0\le x\le a,\;0\le y\le b$ 所确定。
    \end{bbox}

    \begin{bbox}[big]
        \qitem 计算 $\displaystyle I=\iiint_{\Omega}(x^{2}+y^{2})\,\mathrm{d}v$,其中 $\Omega$ 是由 $z=\sqrt{x^{2}+y^{2}},z=1$ 所围区域。
    \end{bbox}

    \begin{bbox}[big]
        \qitem 计算 $\displaystyle I=\iint_{D}(x^{2}-2\sin x+3y+4)\,\mathrm{d}\sigma$,其中 $D: x^{2}+y^{2}\le R^{2}$。
    \end{bbox}

    \begin{bbox}[big]
        \qitem 计算 $\displaystyle I=\iiint_{\Omega}z\sqrt{x^{2}+y^{2}+z^{2}}\,\mathrm{d}v$,其中 $\Omega$ 是由 $z=\sqrt{3(x^{2}+y^{2})},\;x^{2}+y^{2}+z^{2}=4$ 所围成的闭域。
    \end{bbox}

    \begin{bbox}[big]
        \qitem 交换积分次序 $I=\displaystyle \int_{0}^{1}\mathrm{d}x\int_{x}^{\sqrt{2x-x^{2}}}f(x,y)\,\mathrm{d}y$(设 $f(x,y)$ 为连续函数)。
    \end{bbox}

    \begin{bbox}[big]
        \qitem 求由曲面 $z=2x^{2}+y^{2}$ 与 $z=6-x^{2}-2y^{2}$ 所围立体的体积。
    \end{bbox}

\end{qitems}

\section{曲线积分与曲面积分}

\subsection{对弧长的曲线积分}
\begin{qitems}

    \begin{bbox}[choice]
        \qitem 设 $L$ 为 $y=x^{3}$ 与 $y=x$ 所围成区域的整个边界曲线,$f(x,y)$ 是连续函数,则 $\displaystyle \oint_{L}f(x,y)\,\mathrm{d}s=$\blankbox。
        \vspace{1.5mm}

        {\renewcommand{\arraystretch}{1.8}
        \fourchoices
          {$\displaystyle \int_{0}^{1}f(x,x^{3})\,\mathrm{d}x+\int_{0}^{1}f(x,x)\,\mathrm{d}x$}
          {$\displaystyle \int_{0}^{1}f(x,x^{3})\,\mathrm{d}x+\sqrt{2}\int_{0}^{1}f(x,x)\,\mathrm{d}x$}
          {$\displaystyle \int_{-1}^{1}f(x,x^{3})\sqrt{1+9x^{4}}\,\mathrm{d}x+\int_{1}^{-1}f(x,x)\sqrt{2}\,\mathrm{d}x$}
          {$\displaystyle \int_{-1}^{1}\left[f(x,x^{3})\sqrt{1+9x^{4}}+\sqrt{2}f(x,x)\right]\mathrm{d}x$}
        }
    \end{bbox}

    \begin{bbox}[blank]
        \qitem 设 $L$ 为直线 $2x+3y-6=0$ 在第一象限的部分,则 $\displaystyle \int_{L}\ln\left( 8+4x+6y\right)\,\mathrm{d}s=$\blankline。
    \end{bbox}

    \begin{bbox}[big]
        \qitem 设 $L$ 为椭圆 $\dfrac{x^{2}}{4}+\dfrac{y^{2}}{3}=1$,其周长记为 $a$,计算 $\displaystyle \oint_{L}\left( 2xy+3x^{2}+4y^{2}\right)\mathrm{d}s$。
    \end{bbox}

    \begin{bbox}[big]
        \qitem 计算 $\displaystyle \int_{L}\sqrt{x^{2}+y^{2}}\,\mathrm{d}s$,其中 $L$ 分别为 $x=\sqrt{a^{2}-y^{2}}$,$y=x$ 及 $y=0$ 围成的闭曲线。
    \end{bbox}

    \begin{bbox}[big]
        \qitem 设有物质曲线 $L$,极坐标下其方程为 $\rho=e^{2\theta}$,$L$ 上任一点处的线密度 $\mu=\theta$,求从 $\theta=0$ 到 $\theta=2\pi$ 的曲线段的质量。
    \end{bbox}

\end{qitems}

\subsection{对坐标的曲线积分}
\begin{qitems}

    \begin{bbox}[big]
        \qitem (1) 若 $L$ 为抛物线 $y=x^{2}$ 上从点 $(1,1)$ 到点 $(2,4)$ 的一段弧,则
        $$\displaystyle \int_{L}(x+y)\,\mathrm{d}x+(y-x)\,\mathrm{d}y=\blankline;$$
        (2) 若 $L$ 为从点 $(1,1)$ 到点 $(2,4)$ 的直线段,则 $\displaystyle \int_{L}(x+y)\,\mathrm{d}x+(y-x)\,\mathrm{d}y=\blankline$;\vspace{3mm}\\
        (3) 若 $L$ 为先沿直线从点 $(1,1)$ 到点 $(1,4)$,再沿直线从点 $(1,4)$ 到点 $(2,4)$ 的折线,则 $$\displaystyle \int_{L}(x+y)\,\mathrm{d}x+(y-x)\,\mathrm{d}y=\blankline .$$

    \end{bbox}

    \begin{bbox}[choice]
        \qitem 设 $L$ 为自 $t=0$ 到 $t=\pi$ 的曲线 $x=a\cos t,y=a\sin t,z=bt$,则 $\displaystyle \int_{L}y\,\mathrm{d}x-x\,\mathrm{d}y+x\,\mathrm{d}z=$\blankbox。
        \fourchoices{$a^{2}\pi$}{$2a^{2}\pi$}{$-a^{2}\pi$}{$\dfrac{a^{2}}{2}\pi$}
    \end{bbox}

\end{qitems}

\subsection{格林公式及其应用}
\begin{qitems}

    \begin{bbox}[blank]
        \qitem 设 $L$ 为单位圆周上自点 $A(1,0)$ 到点 $B(0,1)$ 在第一象限的一段弧,则 $\displaystyle \int_{L}x\ln(1+x^{2}+y^{2})\,\mathrm{d}x+y\ln(1+x^{2}+y^{2})\,\mathrm{d}y=$\blankline。
    \end{bbox}

    \begin{bbox}[big]
        \qitem 设 $L$ 是三顶点分别为 $(0,0)$、$(3,0)$、$(3,2)$ 的三角形正向边界,则 $\displaystyle \oint_{L}(2x-y+4)\,\mathrm{d}x+(5y+3x-6)\,\mathrm{d}y=$\blankline。
    \end{bbox}

    \begin{bbox}[choice]
        \qitem 设 $L$ 是圆周 $x^{2}+y^{2}=a^{2}$($a>0$)负向一周,则曲线积分 $\displaystyle \oint_{L}(x^{3}-x^{2}y)\,\mathrm{d}x+(xy^{2}-y^{3})\,\mathrm{d}y=$\blankbox。
    \fourchoices{$-\dfrac{\pi a^{4}}{2}$}{$-\pi a^{4}$}{$\pi a^{4}$}{$\dfrac{\pi a^{3}}{3}$}

    \end{bbox}

    \begin{bbox}[big]
    \begingroup
    \linespread{2.2}\selectfont 
        \qitem 计算 $\displaystyle \oint_{L}\frac{-y\,\mathrm{d}x+x\,\mathrm{d}y}{|x|+|y|}$,其中 $L$ 为以点 $A(1,0),B(0,1),C(-1,0),D(0,-1)$ 为顶点的正方形 $ABCD$。
    \par
    \endgroup
    \end{bbox}

    \begin{bbox}[big]
    \begingroup
    \linespread{2.2}\selectfont 
        \qitem 计算 $\displaystyle \int_{L}(x^{2}-y)\,\mathrm{d}x-\left( x+\sin^{2}y\right)\,\mathrm{d}y$,$L$ 是圆周 $x^{2}+y^{2}=2x$ 上由点 $O(0,0)$ 到 $A(1,1)$ 的一段弧。
    \par
    \endgroup
    \end{bbox}
\newpage
    \begin{bbox}[big]
        \qitem 计算 $\displaystyle \oint_{L}\frac{y\,\mathrm{d}x-x\,\mathrm{d}y}{2(x^{2}+y^{2})}$,其中 $L$ 是圆周 $(x-1)^{2}+y^{2}=4$,方向为逆时针方向。
    \end{bbox}
\vspace{2.5cm}
    \begin{bbox}[big]
        \qitem 求 $a,b$,使得曲线积分 $\displaystyle \int_{L}(axy^{2}-y^{3})\,\mathrm{d}x+(6x^{2}y-bxy^{2})\,\mathrm{d}y$ 在整个 $xoy$ 面内与积分路径无关,并计算 $\displaystyle \int_{(1,2)}^{(3,4)}(axy^{2}-y^{3})\,\mathrm{d}x+(6x^{2}y-bxy^{2})\,\mathrm{d}y$ 的值。
    \end{bbox}

\end{qitems}

\subsection{对面积的曲面积分}
\begin{qitems}

    \begin{bbox}[choice]
        \qitem 设 $\Sigma$ 为曲面 $z=2-(x^{2}+y^{2})$ 在 $xoy$ 平面上方部分,则 $\displaystyle \iint_{\Sigma}\mathrm{d}S=$\blankbox。
        \vspace{1.5mm}

        {\renewcommand{\arraystretch}{1.8}
        \fourchoices
          {$\displaystyle \int_{0}^{2\pi}\mathrm{d}\theta\int_{0}^{\rho}\sqrt{1+4\rho^{2}}\;\rho\,\mathrm{d}\rho$}
          {$\displaystyle \int_{0}^{2\pi}\mathrm{d}\theta\int_{0}^{2}\sqrt{1+4\rho^{2}}\;\rho\,\mathrm{d}\rho$}
          {$\displaystyle \int_{0}^{2\pi}\mathrm{d}\theta\int_{0}^{2}(2-\rho^{2})\sqrt{1+4\rho^{2}}\;\rho\,\mathrm{d}\rho$}
          {$\displaystyle \int_{0}^{2\pi}\mathrm{d}\theta\int_{0}^{\sqrt{2}}\sqrt{1+4\rho^{2}}\;\rho\,\mathrm{d}\rho$}
        }
    \end{bbox}
\vspace{3mm}
    \begin{bbox}[choice]
        \qitem 设 $\Sigma$ 是上半球面 $x^{2}+y^{2}+z^{2}=R^{2},\;z\ge 0$,$\Sigma_{1}$ 是 $\Sigma$ 在第一卦限中的部分,则\blankbox。
        \vspace{1.5mm}

        {\renewcommand{\arraystretch}{1.8}
        \fourchoices
          {$\displaystyle \iint_{\Sigma}x\,\mathrm{d}S=4\iint_{\Sigma_{1}}x\,\mathrm{d}S$}
          {$\displaystyle \iint_{\Sigma}y\,\mathrm{d}S=4\iint_{\Sigma_{1}}x\,\mathrm{d}S$}
          {$\displaystyle \iint_{\Sigma}z\,\mathrm{d}S=4\iint_{\Sigma_{1}}x\,\mathrm{d}S$}
          {$\displaystyle \iint_{\Sigma}xyz\,\mathrm{d}S=4\iint_{\Sigma_{1}}xyz\,\mathrm{d}S$}
        }
    \end{bbox}
\vspace{3mm}
    \begin{bbox}[big]
        \qitem 设 $\Sigma$ 是曲面 $z=\dfrac{1}{3}(x^{2}+y^{2})$ 中介于 $z=0$ 及 $z=2$ 之间的部分曲面,求 $\displaystyle \iint_{\Sigma}\frac{z}{\sqrt{9+4x^{2}+4y^{2}}}\,\mathrm{d}S$。
    \end{bbox}

    \begin{bbox}[big]
        \qitem 计算 $\displaystyle \iint_{\Sigma}z^{2}\,\mathrm{d}S$,其中 $\Sigma$ 是 $x^{2}+y^{2}+z^{2}=R^{2}$ 在第一卦限的部分。
    \end{bbox}

    \begin{bbox}[big]
        \qitem 计算 $\displaystyle \oiint_{x^{2}+y^{2}+z^{2}=1}(ax^{2}+by^{2}+cz^{2})\,\mathrm{d}S$。
    \end{bbox}

    \begin{bbox}[big]
        \qitem 计算 $\displaystyle \iint_{\Sigma}\frac{e^{z}}{\sqrt{x^{2}+y^{2}}}\,\mathrm{d}S$,其中 $\Sigma: z=\sqrt{x^{2}+y^{2}}$ 被 $z=1,\;z=2$ 截得部分。
    \end{bbox}

\end{qitems}

\subsection{对坐标的曲面积分}
\begin{qitems}

    \begin{bbox}[blank]
        \qitem $\Sigma$ 为曲面 $x^{2}+y^{2}+z^{2}=a^{2}$ 的外侧,求 $\displaystyle \iint_{\Sigma}x^{2}\,\mathrm{d}y\mathrm{d}z=$\blankline。
    \end{bbox}

    \begin{bbox}[blank]
        \qitem $\Sigma$ 是平面 $x+y+z=1$ 被坐标平面所截得的三角形的上侧,则 $\displaystyle \iint_{\Sigma}x\,\mathrm{d}y\mathrm{d}z+y\,\mathrm{d}z\mathrm{d}x+z\,\mathrm{d}x\mathrm{d}y=$\blankline。
    \end{bbox}

    \begin{bbox}[big]
        \qitem 计算 $\displaystyle \iint_{\Sigma}(x^{2}+y^{2}+z^{2})\sqrt{x^{2}+y^{2}}\,\mathrm{d}x\mathrm{d}y$,$\Sigma$ 是下半球面 $z=-\sqrt{1-x^{2}-y^{2}}$ 的下侧。
    \end{bbox}

    \begin{bbox}[big]
        \qitem 计算 $$\displaystyle \iint_{\Sigma}\left[f(x,y,z)+x\right]\mathrm{d}y\mathrm{d}z+\left[2f(x,y,z)+y\right]\mathrm{d}z\mathrm{d}x+\left[f(x,y,z)+z\right]\mathrm{d}x\mathrm{d}y ,$$其中 $f(x,y,z)$ 为连续函数,$\Sigma$ 为平面 $x-y+z=1$ 在第四卦限部分的上侧。
    \end{bbox}

\end{qitems}

\subsection{高斯公式 通量与散度}
\begin{qitems}

    \begin{bbox}[choice]
        \qitem 设 $\Sigma$ 是球面 $x^{2}+y^{2}+z^{2}=a^{2}$ 的内侧,则 $\displaystyle \iint_{\Sigma}z\,\mathrm{d}x\mathrm{d}y=$\blankbox。
        \fourchoices{$-\dfrac{4}{3}\pi a^{3}$}{$0$}{$\dfrac{4}{3}\pi a^{3}$}{$-4\pi a^{3}$}
    \end{bbox}

    \begin{bbox}[choice]
        \qitem 设 $\Sigma$ 为由 $z=\sqrt{x^{2}+y^{2}}$ 与 $z=\sqrt{2-x^{2}-y^{2}}$ 所围立体的表面外侧,则积分 $\displaystyle \iint_{\Sigma}x^{2}\,\mathrm{d}y\mathrm{d}z+y^{2}\,\mathrm{d}z\mathrm{d}x+z^{2}\,\mathrm{d}x\mathrm{d}y=$\blankbox。
        \fourchoices{}{$\dfrac{\pi}{2}$}{$\pi$}{$2\pi$}
    \end{bbox}

    \begin{bbox}[big]
        \qitem 计算 $$\displaystyle \oiint_{\Sigma}xz\,\mathrm{d}x\mathrm{d}y+xy\,\mathrm{d}y\mathrm{d}z+yz\,\mathrm{d}z\mathrm{d}x ,$$其中 $\Sigma$ 是平面 $x=0,y=0,z=0,x+y+z=2$ 所围成的空间区域的整个边界曲面的外侧。
    \end{bbox}

    \begin{bbox}[big]
        \qitem 计算 $$\displaystyle \iint_{\Sigma}xz^{2}\,\mathrm{d}y\mathrm{d}z+(x^{2}y-z^{3})\,\mathrm{d}z\mathrm{d}x+(2xy+y^{2}z)\,\mathrm{d}x\mathrm{d}y ,$$ 其中 $\Sigma$ 是上半球面 $z=\sqrt{a^{2}-x^{2}-y^{2}}$ 的外侧。
    \end{bbox}

    \begin{bbox}[big]
        \qitem 计算 $$\displaystyle \iint_{\Sigma}\left( x^{3}\cos\alpha+y^{3}\cos\beta+z^{3}\cos\gamma\right)\mathrm{d}S ,$$ $\Sigma: z=\sqrt{a^{2}-x^{2}-y^{2}}$ 外侧,$\cos\alpha,\cos\beta,\cos\gamma$ 为外法线方向余弦。
    \end{bbox}

\end{qitems}

\subsection{斯托克斯公式 环流量与旋度}
\begin{qitems}

    \begin{bbox}[choice]
        \qitem 若 $\Gamma$ 是曲线 $\left\{\begin{array}{l} x^{2}+y^{2}=1,\\ x-y+z=2,\end{array}\right.$ 从 $z$ 轴正向看去取顺时针方向,则曲线积分\newline $\displaystyle I=\oint_{\Gamma}(z-y)\,\mathrm{d}x+(x-z)\,\mathrm{d}y+(x-y)\,\mathrm{d}z=$\blankbox。
        \fourchoices{$2\pi$}{$0$}{$-2\pi$}{$-\pi$}
    \end{bbox}

    \begin{bbox}[choice]
        \qitem 设 $\vec{a}=y\vec{i}+z\vec{j}+x\vec{k}$,则 $\operatorname{rot}\vec{a}=$\blankbox。
        \fourchoices{$\vec{i}+\vec{j}+\vec{k}$}{$-\left( \vec{i}+\vec{j}+\vec{k}\right)$}{$\vec{i}-\vec{j}+\vec{k}$}{$\vec{i}-\vec{j}-\vec{k}$}
    \end{bbox}

    \begin{bbox}[big]
        \qitem 用斯托克斯公式计算 $$ \oint_{L}y\,\mathrm{d}x+3z\,\mathrm{d}y+2x\,\mathrm{d}z ,$$ 其中 $L$ 为圆周 $x^{2}+y^{2}+z^{2}=4,\;x+y+z=0$,从 $y$ 轴正向看去该圆周为逆时针方向。
    \end{bbox}

\end{qitems}

\subsection{本章综合测验}

\textbf{一、选择题}
\begin{qitems}

    \begin{bbox}[choice]
        \qitem 设 $C$ 是从 $A(1,1)$ 到 $B(2,3)$ 的直线,则 $\displaystyle \int_{C}(x+3y)\,\mathrm{d}x+(y+3x)\,\mathrm{d}y=$\blankbox。
        \vspace{1.5mm}

        {\renewcommand{\arraystretch}{1.8}
        \fourchoices
          {$\displaystyle \int_{1}^{2}\left[(x+2x-1)+(2x-1+3x)\right]\mathrm{d}x$}
          {$\displaystyle \int_{1}^{2}\left[(x+2x+1)+(2x-1+3x)\right]\mathrm{d}x$}
          {$\displaystyle \int_{1}^{2}(x+3)\,\mathrm{d}x+\int_{1}^{3}(y-6)\,\mathrm{d}y$}
          {$\displaystyle \int_{1}^{2}(x+3)\,\mathrm{d}x+\int_{1}^{3}(y+6)\,\mathrm{d}y$}
        }
    \end{bbox}
\vspace{3mm}
    \begin{bbox}[choice]
        \qitem 设 $L$ 为 $x^{2}+y^{2}=1$ 正向一周,则 $\displaystyle \oint_{L}e^{y^{2}}\,\mathrm{d}x=$\blankbox。
        \fourchoices{$\pi a^{2}$}{$0$}{$\dfrac{\pi a^{4}}{2}$}{$\dfrac{\pi a^{4}}{4}$}
    \end{bbox}

    \begin{bbox}[choice]
        \qitem 设 $L$ 为光滑的简单闭曲线,并取顺时针方向,则 $L$ 所围区域 $D$ 的面积可表达为\blankbox。
        \vspace{1.5mm}

        {\renewcommand{\arraystretch}{1.8}
        \fourchoices
          {$\displaystyle \oint_{L}y\,\mathrm{d}x$}
          {$\displaystyle \oint_{L}x\,\mathrm{d}y$}
          {$\displaystyle \oint_{L}x\,\mathrm{d}y+y\,\mathrm{d}x$}
          {$\displaystyle \oint_{L}x\,\mathrm{d}x+y\,\mathrm{d}y$}
        }
    \end{bbox}
\vspace{3mm}
    \begin{bbox}[choice]
        \qitem 设 $C$ 表示椭圆 $\dfrac{x^{2}}{a^{2}}+\dfrac{y^{2}}{b^{2}}=1$,其方向为顺时针方向,则曲线积分 $\displaystyle \oint_{L}(x^{2}+y)\,\mathrm{d}x=$\blankbox。
        \fourchoices{$\pi ab$}{$\pi a^{2}b$}{$a+b^{2}$}{$-\pi ab$}
    \end{bbox}

    \begin{bbox}[choice]
        \qitem 设 $L$ 为圆周 $x^{2}+y^{2}=1$,则 $\displaystyle \oint_{L}x^{2}\,\mathrm{d}s=$\blankbox。
        \fourchoices{$\pi$}{$0$}{$-\pi$}{$\dfrac{\pi}{2}$}
    \end{bbox}

    \begin{bbox}[choice]
        \qitem 设 $\Sigma$ 为球面 $x^{2}+y^{2}+z^{2}=R^{2}$ 的下半球面下侧,则 $\displaystyle \iint_{\Sigma}z\,\mathrm{d}x\mathrm{d}y=$\blankbox。
        \vspace{1.5mm}

        {\renewcommand{\arraystretch}{1.8}
        \fourchoices
          {$-\displaystyle \int_{0}^{2\pi}\mathrm{d}\theta\int_{0}^{R}\sqrt{R^{2}-r^{2}}\,\mathrm{d}r$}
          {$\displaystyle \int_{0}^{2\pi}\mathrm{d}\theta\int_{0}^{R}\sqrt{R^{2}-r^{2}}\,r\,\mathrm{d}r$}
          {$-\displaystyle \int_{0}^{2\pi}\mathrm{d}\theta\int_{0}^{R}\sqrt{R^{2}-r^{2}}\,r\,\mathrm{d}r$}
          {$\displaystyle \int_{0}^{2\pi}\mathrm{d}\theta\int_{0}^{R}\sqrt{R^{2}-r^{2}}\,\mathrm{d}r$}
        }
    \end{bbox}

\end{qitems}
\newpage
\textbf{二、填空题}
\vspace{3mm}
\begin{qitems}

    \begin{bbox}[blank]
        \qitem $L:2x+y=2$ 在第一象限部分,则 $\displaystyle \int_{L}(6-2x-y)\,\mathrm{d}s=$\blankline。
    \end{bbox}

    \begin{bbox}[blank]
        \qitem 设 $L$ 是抛物线 $x=y^{2}$ 上由点 $A(4,2)$ 到点 $B(4,-2)$ 的一段弧,则 $\displaystyle \int_{L}2xy\,\mathrm{d}x+x^{2}\,\mathrm{d}y=$\blankline。
    \end{bbox}

    \begin{bbox}[blank]
        \qitem $\Sigma$ 为柱面 $x^{2}+y^{2}=R^{2}$ 介于 $z=0,z=H$ 之间的部分,则 $\displaystyle \iint_{\Sigma}(x^{2}+y^{2})\,\mathrm{d}S=$\blankline。
    \end{bbox}

    \begin{bbox}[blank]
        \qitem $\Sigma$ 为曲面 $x^{2}+y^{2}+z^{2}=a^{2}$ 的外侧,求 $\displaystyle \iint_{\Sigma}y^{2}\,\mathrm{d}x\mathrm{d}z=$\blankline。
    \end{bbox}

    \begin{bbox}[blank]
        \qitem $\Sigma$ 为曲面 $x^{2}+y^{2}+z^{2}=a^{2}$ 第一卦限,求 $\displaystyle \iint_{\Sigma}y^{2}\,\mathrm{d}S=$\blankline。
    \end{bbox}

\end{qitems}
\vspace{3mm}
\textbf{三、计算题}
\vspace{3mm}
\begin{qitems}

    \begin{bbox}[big]
        \qitem 计算 $\displaystyle \int_{L}\sqrt{x^{2}+y^{2}}\,\mathrm{d}s$,其中 $L$ 为 $x^{2}+y^{2}=ay$ 的整个圆周。
    \end{bbox}

    \begin{bbox}[big]
        \qitem 计算曲面积分 $\displaystyle \iint_{\Sigma}(x^{2}+y^{2})z\,\mathrm{d}S$,其中 $\Sigma$ 是上半球面 $x^{2}+y^{2}+z^{2}=a^{2}$($a>0$)。
    \end{bbox}

    \begin{bbox}[big]
        \qitem 计算 $\displaystyle \int_{L}(-y\,\mathrm{d}x+x\,\mathrm{d}y)$,$L$ 是沿曲线 $y=\sqrt{2x-x^{2}}$ 从点 $A(2,0)$ 到 $O(0,0)$ 的有向弧段。
    \end{bbox}

    \begin{bbox}[big]
        \qitem 计算 $$\displaystyle \oiint_{\Sigma}\frac{x}{r^{3}}\,\mathrm{d}y\mathrm{d}z+\frac{y}{r^{3}}\,\mathrm{d}z\mathrm{d}x+\frac{z}{r^{3}}\,\mathrm{d}x\mathrm{d}y ,$$ 其中 $r=\sqrt{x^{2}+y^{2}+z^{2}}$,$\Sigma$ 为球面 $x^{2}+y^{2}+z^{2}=a^{2}$ 的内侧。
    \end{bbox}

    \begin{bbox}[big]
        \qitem 计算 \[ \displaystyle \iint_{\Sigma}(x^{3}\cos\alpha+y^{3}\cos\beta+z^{3}\cos\gamma)\,\mathrm{d}S ,\]  $\Sigma: z=\sqrt{a^{2}-x^{2}-y^{2}}$外侧,$\cos\alpha,\cos\beta,\cos\gamma$ 为外法线方向余弦。
    \end{bbox}

    \begin{bbox}[big]
        \qitem 求常数 $k$,使曲线积分 $\displaystyle \int_{L}(3x^{2}+y^{2})\,\mathrm{d}x+kxy\,\mathrm{d}y$ 在整个 $xoy$ 面内与积分路径无关,并计算 $\displaystyle \int_{(0,1)}^{(2,2)}(3x^{2}+y^{2})\,\mathrm{d}x+kxy\,\mathrm{d}y$ 的值。
    \end{bbox}

\end{qitems}

\end{document}
